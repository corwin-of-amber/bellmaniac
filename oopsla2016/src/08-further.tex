\medskip\cbstart
\section{Further Improvements}
\label{further}

\diffnote{section added in response to questions about possible extensions}%
Overall usability can be greatly enhanced with some syntactic sugar
for program terms and tactic application commands.
The current syntax is part of a proof-of-concept UI.
User experience can also be enriched with direct manipulation gestures,
such as pointing to a sub-term in the program rather than using boxed
literals to refer to them, and ``carving'' sub-computations from the
current term by selecting and dragging.

The tactic library can be extended with a number of important optimization
for this domain. \newterm{Dimensionality reduction} is a technique that
allows to lower an $n$-dimensional DP problem to $n-1$ dimensions by
computing layer by layer.
This has an acute difference from merely buffering the most recent layer;
without going too much into details, we point out that intermediate layers
computed this way may differ from the corresponding projection of the
$n$-dimensional table, while the final layer is identical.
This tactic can enable a whole sub-class of high dimension DP problems.

Some low-level tactics can boost performance as well: the Protein benchmark
runs faster using the hand-coded version due to a subtle case splitting
around the innermost loop, that reduces an expression of the form
$min(k,2j-i+1)$ to just $2j-i+1$, allowing in turn to factor an addend
out of the loop ($k$ is the innermost loop variable).
Doing the same optimization by hand on the code produced by Bellmania leads
to the same speedup, but this boost cannot be claimed until it is automated.

Z-ordering of the array is also a general technique and can be automated
as part of the compiler back-end. However, this and other optimizations (such
as vectorization and copy optimization mentioned in \Cref{codegen}) are
delicate and sometimes even degrade performance, so they better be auto-tuned
rather than applied heuristically.
\cbend