\documentclass{article}

\usepackage[letterpaper, margin=1in]{geometry}
\pagenumbering{gobble}

\begin{document}

\section*{Deriving Divide-and-Conquer Dynamic Programming Algorithms Using 
  Solver-Aided Transformations}
\subsection*{Preface for revised version}

\hrule

\bigskip

We are pleased to provide you with the updated version of our submission to the OOPSLA track at SPLASH~2016,
containing required changes and revisions.

\bigskip
Following is a list of requests that were addressed in this revision.

\begin{enumerate}
  \item To clarify the user interaction workflow and prepare the reader
    for what is to come, we added a subsection ``Interaction Model''
    following the design overview (end of Section 2).
    The last paragraph outlines the next two sections.
  \item Per the request of the reviewers, we separated implementation details
    from the result of the experiments (Section 7).
  \item To make for a more appealing evaluation, we refined the description
    of our chosen metric for usability. We made clear that we admit it is
    a crude proxy and is no substitute for a more disciplined user study.
  \item To make it more in line with the rest of the paper, we changed
    the narrative in the conclusion (Section 10).
  \item We expanded the discussion of refinement types and refinement rules
    (Section 3.4) explaining why the format of droplet was chosen.
  \item We filled in missing details that were unclear to the reviewers,
    such as the used of boxed numerals to designate sub-terms, the notorious
    ``Obligations: tactic,'' and where obligations come from.
  \item To make up for the terseness of formal definitions of tactics (Section 4),
    esp. \textsf{Stratify}, we expanded the discussion of it in text and
    added an example and a small illustration.
  \item We extended our comparison with Autogen (Section 9). We hope that
    the benefits of a deductive approach are clearer now.
  \item To answer questions about possible extensions, we added a section
    titled Further Improvements (Section 8), and also expanded the conclusion
    with some future outlook.
  \item To address the questions about certification, we made it clear in the
    design overview (Section 2) that the system verifies every step using SMT,
    but does not generate machine-checkable proofs in the style of Coq
    (although this should be conceptually easy, assuming SMT-generated proofs are themselves checkable).
  \item Additional references pointed out by the reviewers have been incorporated as
    related work (Section 9).
\end{enumerate}

We marked the revised version with change bars and margin notes to highlight
the location of the above changes and a few smaller ones.
Since a \texttt{latexdiff} is virtually impossible to obtain for any nontrivial \TeX{} document,
we attach the source diff in hope that it will complement the inline markup.

\bigskip\noindent
Regards,\\
the authors
\end{document}