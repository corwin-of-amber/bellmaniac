\section{Conclusion}
\label{conc}

\cbstart\diffnote{\vspace{0mm}Completely changed the narrative, and alluded to applicability in different domains as future work}%
The examples in this paper show that a few well-placed tactics can cover a wide range
of program transformations. The introduction of solver-aided tactics allowed us to make
the library of tactics smaller, by enabling the design of higher-level, more generic
tactics. Their small number gives the hope that end-users with some mathematical background
will be able to use the system without the steep learning curve that is usually associated
with proof assistants. This can be a valuable tool for algorithms research.

Moreover, limiting the number of tactics shrinks the space in which to search for programs,
so that an additional level automation may be achieved via AI or ML methods. As more
developments are done by humans and collected in a database, those algorithms would become
more adept in predicting the next step of the construction.

In a broader outlook, the technique for formalizing transformation tactics is not
particularly attached to divide-and-conquer algorithms and their implementations.
In this work, we constructed a generic tactic application engine, on top of which
the tactics required for our domain were easy to implement.
This gives rise to the hope that, in the future, the same approach can be applied
to other domains, in the interest of encoding domain knowledge, providing better
DSLs that offer users the power to write high-level programs without sacrificing
performance.
\cbend

\section*{Acknowledgments}

This work is supported by NSF Grants CCF-1139056, CCF-1439084 and CNS-1553510. We thank Shoaib Kamil and Adobe research for their feedback and support.
