\section{Overview}
\label{overview}

Most readers are likely familiar with the Dynamic Programming (DP) technique of Richard Bellman~\cite{03/Bellman:DP} to construct an optimal solution to a problem by combining together optimal solutions to many overlapping sub-problems. The key to DP is to exploit the overlap and reuse computed values to explore exponential-sized solution spaces in polynomial time. Dynamic programs are usually described through recurrence relations that specify how to decompose sub-problems, and is typically implemented using a DP table where each cell holds the computed solution for one of these sub-problems. The table can be filled by visiting each cell once in some predetermined order, but recent research has shown that it is possible to achieve order-of-magnitude performance improvements over this standard implementation approach by developing \emph{divide-and-conquer}  implementation strategies that recursively
partition the space of subproblems into smaller subspaces~\cite{IPDPS15/Tithi,SODA14/Bender,SODA06/Chowdhury,SPAA08/Chowdhury,TOCS10/Chowdhury,TCBB10/Chowdhury}. 

Before delving into how Bellmania supports the process of generating such an implementation, it is useful to understand how a traditional iterative implementation works. For this, we will use the 
optimal parenthesization algorithm from the introduction (\Cref{intro:naive}). The problem description is as follows: given a sequence of factors $a_0\!\cdots a_{n-1}$, 
the goal is to discover a minimal-cost placement of parentheses in the product expression
assuming that multiplication is associative but not commutative. The cost
of reading $a_i$ is given by $x_i$, and that the cost for multiplying
$\Pi(a_{i..(k-1)})$ by $\Pi(a_{k..(j-1)})$ is given by $w_{ikj}$. The specification of the algorithm is shown in \Cref{overview:paren spec}; 
the values $x_i$ and  $w_{ikj}$ are inputs to the algorithm, and the output is a table $G$, where each element $G_{ij}$ is the lowest cost for parenthesizing $a_{i..(j-1)}$, with $G_{0n}$ being the overall optimum.

\begin{figure}[b]
\begin{tabular}{@{\hspace{-4pt}}r@{~~~}p{5cm}@{\hspace{-4pt}}}
\begin{tikzpicture}[x=4.1mm,y=4.1mm,baseline=(center), remember picture]
  \coordinate(center) at (3,3);
  \def\n{7}  % actually n+1 but let's not be nitpicky
  \draw[black!25!white] (\n,1) |- (0,0) |- (1,\n);
  \draw (\n,1) -- (\n,\n) -- (1,\n);
  \foreach \i [evaluate=\i] in {2-1,...-1,\n-1} {
    \draw[black!25!white] (0,\i) -- (\n-\i,\i); 
    \draw[black!25!white] (\i,0) -- (\i,\n-\i); 
    \draw (\n,\i) -- (\n-\i,\i); 
    \draw (\i,\n) -- (\i,\n-\i);
  }
  %\draw[step=1] (0,0) grid (6,6);
  \draw[ultra thick] (5,5) rectangle +(1,1);
  \node[circle,fill=BrickRed,inner sep=0,minimum size=1mm](Gij) at (5.5,5.5) {};
  \fill[black,opacity=0.1] (1,\n) rectangle ++(1,-1) rectangle ++(1,-1) rectangle ++(1,-1) rectangle ++(1,-1) rectangle ++(1,-1) rectangle ++(1,-1);
  \fill[blue,opacity=0.2] (2,5) rectangle (5,6);
  \fill[blue,opacity=0.2] (5,2) rectangle (6,5);
  \node[anchor=south east](G) at (0,\n) {\small$G$};
  \draw[->] (G.east) -- +(1.5,0) node[anchor=west] {\small $j$};
  \draw[->] (G.south) -- +(0,-1.5) node[anchor=north] {\small $i$};
\end{tikzpicture}
&
\vspace{-1.5cm}
\small
$
\begin{array}{@{}l@{}}
	\tikz[overlay, remember picture]{\draw[BrickRed] (-.5mm,0) -- (Gij);}
	G_{ij} ~=~ \\
	~
	\begin{cases}
		~~x_i                        & i{+}1=j \\
	    \underset{i<k<j}\min ~ G_{ik} {+} G_{kj} {+} w_{ikj} & i{+}1<j
	\end{cases}
\end{array}
$

\vspace{6mm}
% legend
\cbstart
\begin{tikzpicture}
  \node(base)[draw,fill=black!10!white,label=right:{\fontsize{8}{8}\selectfont base case}] {};
  \node(dep)[right=1.5cm of base,draw,fill=blue!20!white,label=right:{\fontsize{8}{8}\selectfont dependencies}] {};
  \node(emp)[right=1.8cm of dep,draw=black!25!white,label=right:{\fontsize{8}{8}\selectfont empty}] {};
\end{tikzpicture}
\cbend
\end{tabular}
\caption{Recurrence equation and cell-level dependencies.}
\label{overview:paren spec}
\end{figure}

\begin{paragraph}{Iterative Algorithm.}
Using the standard dynamic programming method, anyone who has  read \cite{09/CLRS} would compute this recurrence
with an iterative program by understanding the dependency pattern:
to compute the $\min_{i<k<j}(\cdots)$ expression in \Cref{overview:paren spec}
the algorithm needs to enumerate $k$ and gather information from all cells below and to the left of $G_{ij}$.
In particular, each value $G_{ij}$ is computed from other values $G_{i'j'}$ with higher
row indexes $i'>i$ and lower column indexes $j'<j$. 
Therefore, considering $G$ as a two-dimensional array, it can be filled in a single sweep from left to right and from bottom
to top, as done in \Cref{intro:naive}.
\end{paragraph}




\newcommand\qbox[1]{\fbox{\rm\scriptsize#1}}
\newcommand\tinyqbox[1]{\hspace{.5pt}\tikz \node[draw,inner sep=1.5pt] {$\scriptscriptstyle #1$};}

\newcommand\plusoneocd{\raisebox{.5pt}{$\scriptstyle+1$}}

\algrenewtext{Procedure}{\hspace{-3mm}{\bf procedure}~}  % hack to make proc header slighly less indented

\begin{paragraph}{Divide-and-Conquer Algorithm.}
To illustrate the main concepts underlying Bellmania and the key ideas in deriving divide-and-conquer implementations, 
we will walk through the first
few steps that an algorithms expert --- whom we will call Richard --- would follow using Bellmania to generate a provably correct divide-and-conquer implementation of the optimal parenthesization algorithm.

In the Bellmania development model, Richard will start with the specification from \Cref{overview:paren spec}, and progressively manipulate it to get the specification in a form that reflects the recursive structure of the divide-and-conquer implementation. At any step in the transformation, Bellmania can generate code from the partially transformed specification. Code generated from the initial specification will yield an implementation like \Cref{intro:naive}, whereas generating code from the fully transformed specification will yield the divide-and-conquer implementation that we want.
In the rest of the text, we will use the term \newterm{program} to refer to any of the partially transformed specifications.




\begin{figure}
\centering
\begin{tabular}{c@{\hspace{.5in}}c}
\begin{tikzpicture}[baseline=(n/2), q/.style={font=\relsize{1.3}}, pale/.style={color=black!25!white}]
  % this code draws a grid with the bottom-left paler
  %\draw (0,2) -- (2,2) -- (2,0);
  %\foreach \i in {0,1} {
  %  \draw[pale] (\i,0) -- (\i, 1-\i); \draw (\i,2) -- (\i, 1-\i);
  %  \draw[pale] (0,\i) -- (1-\i, \i); \draw (2,\i) -- (1-\i, \i);
  %}
  % but this might be confusing. let's just paint
  % a regular 2x2 grid
  \draw (0,0) grid (2,2);
  \node[q] at (.5,1.5) {1};   \node[q] at (1.5,1.5) {2};
  \node[q] at (.5, .5) {3};   \node[q] at (1.5, .5) {4};
  \node(O)[above left] at (0,2) {$0$};
  \node(m/2)[above] at (1,2) {$\frac{n}{2}$};
  \node(m)[above] at (2,2) {$n$};
  \node(n/2)[left] at (0,1) {$\frac{n}{2}$};
  \node(n)[left] at (0,0) {$n$};
  \node(J0)[above] at (.5,2.5) {$J_0$};
  \node(J1)[above] at (1.5,2.5) {$J_1$};
  \node(I0)[left] at (-.5,1.5) {$J_0$};
  \node(I1)[left] at (-.5,.5) {$J_1$};
  %\coordinate(0) at (0,0);
  %\coordinate(sw) at (0,0);
  %\coordinate(ne) at (2,2);
  %\draw (J0.north -| sw) -- node[above] {$J$} ++(ne |- 0);
  %\draw (I0.west |- sw) -- node[left] {$I$} ++(ne -| 0);
  \draw (O.north east) -- (O.north east -| m/2.north west);
  \draw (O.north east -| m.110) -- (O.north east -| m/2.north east);
  \draw (O.south west) -- (O.south west |- n/2.north west);
  \draw (O.south west |- n.160) -- (O.south west |- n/2.220);
\end{tikzpicture}
& 
$\begin{array}{l}
  \qbox2 \mbox{\textit{ depends on }} \qbox1 \\[2pt]
  \qbox2 \mbox{\textit{ depends on }} \qbox4 \\ 
  \\
  (\,\qbox3\mbox{\textit{ is empty}})
\end{array}$
\end{tabular}
\vspace{5pt}
\caption{\label{overview:quadrants}
  Dividing a two-dimensional array into quadrants; the dependencies for the case of the running example are shown on the right.}
\end{figure}


\Cref{overview:slice-stratify-synth} provides a visual description of the initial stages of the transformation process. The figure includes block diagrams illustrating how the program at a given stage in the transformation will compute its output table from its input. For example, the first row corresponds to the program before any transformations take place, i.e.~the initial specification. At this stage, the program involves a single loop nest that reads from the entire array and writes to the entire array; solid arrows in the diagram denote data dependency. The transformation from one stage to the next, the dashed arrows, is performed by the application of  \newterm{tactics} that represent a high-level refinement concept. 

As a first step in the transformation, Richard would like to partition the two-di\-men\-sio\-nal array $G$ into quadrants, as illustrated in \Cref{overview:quadrants}. In Bellmania, the partition is accomplished by applying the {\sf Slice} tactic, illustrated graphically at the top of \Cref{overview:slice-stratify-synth}. In order to escape the need to reason about concrete array indices, Bellmania provides an abstract view where
the partitions are labeled $J_0,J_1$. The effect of {\sf Slice} is shown in text in \Cref{overview:logical-slice-stratify}$(a)$ --- the figure trades accuracy for succinctness by omitting the base case of the recurrence for now. Due to the structure of the problem, namely $i\!<\!j$, 
the bottom-left quadrant (\qbox3 in \Cref{overview:quadrants}) is empty.
Slicing therefore produces only three partitions.


The computation of \qbox1
(the top-left quadrant)
does not depend on any of the other computations, so Richard applies the
{\sf Stratify} tactic, which separates an independent computation step as a separate loop. This is equivalent to rewriting the specification as in \Cref{overview:logical-slice-stratify}({\it b}):
the first computation is given a special name $G^{\tinyqbox1}$, then the following
computations read data either from $G^{\tinyqbox1}$ (when the indices are in \qbox1)
or from $G$ (otherwise), which is denoted by $G^{\tinyqbox1}\!/G$. The ``$/\,$'' operator
is part of the Bellmania language and will be defined formally in \Cref{lang}.
Bellmania checks the data dependencies and verifies that the transformation
is sound.

Repeating {\sf Stratify} results in a three-step computation,
as seen in \Cref{overview:slice-stratify-synth}$(c)$, from which Richard can obtain the program in \Cref{overview:breakdown}.
This already gives some performance gain, since computations of \qbox1 and \qbox4 can now run in parallel.
Bellmania is capable of sound reasoning about parallelism, using knowledge encoded
via types of sub-terms, showing that two threads are race-free when they work on different regions
of the table.
This is handled automatically by the code generator.
\end{paragraph}


\begin{figure}
\vspace{-.5em}
\[\renewcommand\arraystretch{1.3}
  \begin{array}{@{}l@{}}
    \textsf{Slice} \quad i,j:\langle J_0\times J_0 \,|\, J_0\times J_1 \,|\, J_1\times J_1\rangle     \hfill (a)\\
    ~~\forall i,j\in\qbox1.~ G_{ij} = \underset{i<k<j}\min ~ G_{ik} + G_{kj} + w_{ikj} \\
    ~~\forall i,j\in\qbox2.~ G_{ij} = \underset{i<k<j}\min ~ G_{ik} + G_{kj} + w_{ikj} \\
    ~~\forall i,j\in\qbox4.~ G_{ij} = \underset{i<k<j}\min ~ G_{ik} + G_{kj} + w_{ikj} \\
    %
    \textsf{Stratify} ~ \qbox1 \hfill (b)\\
    %
    ~~\forall i,j\in\qbox1.~ G^{\tinyqbox1}_{ij} = \underset{i<k<j}\min ~ G^{\tinyqbox1}_{ik} + G^{\tinyqbox1}_{kj} + w_{ikj} \\
    ~~\forall i,j\in\qbox2.~ G_{ij} = \underset{i<k<j}\min ~ (G^{\tinyqbox1}\!/G)_{ik} + (G^{\tinyqbox1}\!/G)_{kj} + w_{ikj} ~~ \\
    ~~\forall i,j\in\qbox4.~ G_{ij} = \underset{i<k<j}\min ~ (G^{\tinyqbox1}\!/G)_{ik} + (G^{\tinyqbox1}\!/G)_{kj} + w_{ikj} \\
  \end{array}
\]
\vspace{-.75em}
\caption{\label{overview:logical-slice-stratify}
  The first two steps in the development, represented as logical specifications.}
\end{figure}


\newcommand\steparrowwidth{3mm}
\newcommand\steparrow{\includegraphics[width=\steparrowwidth]{img/arrow}}

\begin{algorithm}[b]
\renewcommand\arraystretch{1.3}
\begin{algorithmic}
\Procedure{A{\larger{[}}$G${\larger{]}}}{}\EndProcedure
  \For{$i=(n{-}2)..0 \cap J_0$}    \Head{Compute \qbox1}
    \For{$j=(i{+}2)..n \cap J_0$} 
      \State $G_{ij} :=
          \underset{i<k<j}\min ~ G_{ik} + G_{kj} + w_{ikj}$
    \EndFor
  \EndFor
  \For{$i=(n-2)..0 \cap J_1$}    \Head{Compute \qbox4}
    \For{$j=(i{+}2)..n \cap J_1$} 
      \State $G_{ij} :=
          \underset{i<k<j}\min ~ G_{ik} + G_{kj} + w_{ikj}$
    \EndFor
  \EndFor
  \For{$i=(n{-}2)..0 \cap J_0$}    \Head{Compute \qbox2}
    \For{$j=(i{+}2)..n \cap J_1$}
      \State $G_{ij} :=
          \underset{i<k<j}\min ~ G_{ik} + G_{kj} + w_{ikj}$
    \EndFor
  \EndFor
\end{algorithmic}
\caption[t]{\label{overview:breakdown}
   \cbstart Parenthesis\cbend{} --- Sliced and Stratified}
\end{algorithm}


\newbox\primebox
\setbox\primebox\hbox{$'$}
\newbox\doubleprimebox
\setbox\doubleprimebox\hbox{$''$}

\newcommand\primeocd[1]{\hspace{\wd\primebox}#1\usebox\primebox}
\newcommand\doubleprimeocd[1]{\hspace{\wd\doubleprimebox}#1\usebox\doubleprimebox}

\begin{figure}
\centering
\begin{tikzpicture}[>=latex,x=6mm,y=6mm,
    every path/.style={step=1},
    every node/.style={inner sep=.5pt},
    init data/.style={pattern=north west lines, pattern color=green!50!black},
    block/.style={rectangle,draw,very thick,fill=Orange, fill opacity=0.2, inner sep=0},
    knob/.style={midway,draw,fill=white,midway,inner sep=1.25pt}]
    
  \def\dx{1.75cm}
  \def\dy{1.75cm}
  
  % -------------------------------------------------------

  \draw (0,0) rectangle (2,2);
  \node(s)[block,fit={(0,0) (2,2)}] {};
  \node[inner sep=3mm](0) at (1,1) {0};

  \node[inner sep=0] at (2.5,1) {\steparrow};

  \tikzset{xshift=\dx}

  \draw (0,0) rectangle (2,2);
  \node[inner sep=3mm](0) at (1,1) {0$'$};
  \draw (s.70) edge[->,out=20] (0);

  \applytacticnode{Slice ~~ $i : \langle I_0|I_1\rangle$ ~~ $j : \langle J_0|J_1\rangle$};

  \tikzset{yshift=-\dy,xshift=-\dx}
    
  % -------------------------------------------------------

  \draw (0,0) grid (2,2);
  \node(1) at (.5,1.5) {1};   \node(2) at (1.5,1.5) {2};
  \node(3) at (.5, .5) {3};   \node(4) at (1.5, .5) {4};
  \node(s)[block,fit={(0,0) (2,2)}] {};

  \node[inner sep=0] at (2.5,1) {\steparrow};

  \tikzset{xshift=\dx}

  \draw (0,0) grid (2,2);
  \node(1) at (.5,1.5) {\primeocd 1};   \node(2) at (1.5,1.5) {\primeocd 2};
  \node(3) at (.5, .5) {\primeocd 3};   \node(4) at (1.5, .5) {\primeocd 4};
  \draw (s.70) edge[->,out=20] (1);
  \draw (s.70) edge[->,out=20] (2);
  \draw (s.70) edge[->,out=20] (3);
  \draw (s.70) edge[->,out=20] (4);

  \applytacticnode{Stratify \qbox1};

  % -------------------------------------------------------
  \setlength\baselineskip{15pt}  

  \tikzset{yshift=-\dy,xshift=-\dx}

  \draw (0,0) grid (2,2);
  \node(s)[block,fit={(0,1) (1,2)}] {};
  \node(1) at (.5,1.5) {1};   \node(2) at (1.5,1.5) {2};
  \node(3) at (.5, .5) {3};   \node(4) at (1.5, .5) {4};

  \node[inner sep=0] at (2.5,1) {\steparrow};

  \tikzset{xshift=\dx}
  
  \draw (0,0) grid (2,2);
  \node(1) at (.5,1.5) {\primeocd 1};   \node(2) at (1.5,1.5) {2};
  \node(3) at (.5, .5) {3};             \node(4) at (1.5,.5) {4};
  \draw (s.70) edge[->,out=20] node[knob,circle] {} (1);
  \node(s)[block,fit={(0,0) (2,2)}] {};
  
  \node[inner sep=0] at (2.5,1) {\steparrow};

  \tikzset{xshift=\dx}

  \draw (0,0) grid (2,2);
  \node(1) at (.5,1.5) {\primeocd 1};   \node(2) at (1.5,1.5) {\primeocd 2};
  \node(3) at (.5, .5) {\primeocd 3};   \node(4) at (1.5, .5) {\primeocd 4};
  \draw (s.70) edge[->,out=20] (2);
  \draw (s.70) edge[->,out=20] (3);
  \draw (s.70) edge[->,out=20] (4);
  
  \applytacticnode{Synth $\circ ~~ \dashrightarrow A^{I_0J_0}$ \\ 
      \applytactic{Stratify \qbox2}};
  
  \tikzset{yshift=-\dy,xshift=-2*\dx}

  % -------------------------------------------------------

  \draw (0,0) grid (2,2);
  \node(s)[block,fit={(0,1) (1,2)}] {};
  \node(1) at (.5,1.5) {1};   \node(2) at (1.5,1.5) {2};
  \node(3) at (.5, .5) {3};   \node(4) at (1.5, .5) {4};

  \node[inner sep=0] at (2.5,1) {\steparrow};

  \tikzset{xshift=\dx}
  
  \draw (0,0) grid (2,2);
  \node(1) at (.5,1.5) {\primeocd 1};   \node(2) at (1.5,1.5) {2};
  \node(3) at (.5, .5) {3};   \node(4) at (1.5, .5) {4};
  \draw (s.70) edge[->,out=20] node[knob,circle] {} (1);
  \node(s)[block,fit={(0,1) (2,2)}] {};
  \node[inner sep=0] at (2.5,1) {\steparrow};

  \tikzset{xshift=\dx}
  
  \draw (0,0) grid (2,2);
  \node(1) at (.5,1.5) {\primeocd 1};  \node(2) at (1.5,1.5) {\primeocd 2};
  \node(3) at (.5,.5) {3};             \node(4) at (1.5,.5) {4};
  \draw (s.70) edge[->,out=20] (2);
  \node[inner sep=0] at (2.8,1) {$\cdots$};

  \tikzset{xshift=.25*\dx}

  \node[right,align=left] at (3,1) {\applytactic{Slice \qbox2 \\ \quad $p : \langle J_0|J_1\rangle$}};

  \tikzset{yshift=-\dy,xshift=-2.25*\dx}

  % -------------------------------------------------------

  \draw (0,0) grid (2,2);
  \node(s)[block,fit={(0,1) (1,2)}] {};
  \node(1) at (.5,1.5) {1};   \node(2) at (1.5,1.5) {2};
  \node(3) at (.5, .5) {3};   \node(4) at (1.5, .5) {4};

  \node[inner sep=0] at (2.5,1) {\steparrow};

  \tikzset{xshift=\dx}
  
  \draw (0,0) grid (2,2);
  \node(1) at (.5,1.5) {\primeocd 1};   \node(2) at (1.5,1.5) {2};
  \node(3) at (.5, .5) {3};   \node(4) at (1.5, .5) {4};
  \draw (s.70) edge[->,out=20] node[knob,circle] {} (1);
  \node(s1)[block,fit={(0,1) (1,2)}] {};
  \node(s2)[block,fit={(1,1) (2,2)}] {};
  
  \node[inner sep=0] at (2.5,1) {\steparrow};

  \tikzset{xshift=\dx}
  
  \draw (0,0) grid (2,2);
  \node(1) at (.5,1.5) {\primeocd 1};  \node(2) at (1.5,1.5) {\primeocd 2};
  \node(3) at (.5,.5) {3};             \node(4) at (1.5,.5) {4};
  \draw (s1.70) edge[->,out=20] (2);
  \draw (s2.70) edge[->,out=20] (2);
  \node[inner sep=0] at (2.8,1) {$\cdots$};

  \tikzset{xshift=.25*\dx}

  \node[right,align=left] at (3,1) {\applytactic{Stratify \qbox2$^{\# 1}$}};

  \tikzset{yshift=-\dy,xshift=-2.25*\dx}

  % -------------------------------------------------------

  \draw (0,0) grid (2,2);
  \node(s)[block,fit={(0,1) (1,2)}] {};
  \node(1) at (.5,1.5) {1};   \node(2) at (1.5,1.5) {2};
  \node(3) at (.5, .5) {3};   \node(4) at (1.5, .5) {4};

  \node[inner sep=0] at (2.5,1) {\steparrow};

  \tikzset{xshift=\dx}
  
  \draw (0,0) grid (2,2);
  \node(1) at (.5,1.5) {\primeocd 1};   \node(2) at (1.5,1.5) {2};
  \node(3) at (.5, .5) {3};   \node(4) at (1.5, .5) {4};
  \draw (s.70) edge[->,out=20] node[knob,circle] {} (1);
  \node(s)[block,fit={(0,1) (1,2)}] {};
  
  \node[inner sep=0] at (2.5,1) {\steparrow};

  \tikzset{xshift=\dx}
  
  \draw (0,0) grid (2,2);
  \node(1) at (.5,1.5) {\primeocd 1};  \node(2) at (1.5,1.5) {\primeocd 2};
  \node(3) at (.5,.5) {3};             \node(4) at (1.5,.5) {4};
  \draw (s.70) edge[->,out=20] node[knob,isosceles triangle,rotate=90] {} (2);
  \node(s)[block,fit={(1,1) (2,2)}] {};

  \node[inner sep=0] at (2.5,1) {\steparrow};

  \tikzset{xshift=\dx}
  
  \draw (0,0) grid (2,2);
  \node(1) at (.5,1.5) {\primeocd 1};  \node(2) at (1.5,1.5) {\doubleprimeocd 2};
  \node(3) at (.5,.5) {3};             \node(4) at (1.5,.5) {4};
  \draw (s.70) edge[->,out=20] node[knob,diamond] {} (2);
  \node[inner sep=0] at (2.8,1) {$\cdots$};

  \tikzset{xshift=-1.25*\dx,yshift=-.75*\dy}

  \applytacticnode{Synth {\tikz\node[knob,isosceles triangle,rotate=90] {};} $~~ \longrightarrow B^{I_0J_0J_1}$ \\ 
      \applytactic{Synth $\diamond$} $~~ \dashrightarrow A^{I_0J_1}$};

  \tikzset{yshift=-\dy,xshift=-2*\dx}

\end{tikzpicture}

\caption[caption]{\label{overview:slice-stratify-synth}
  Overview of tactic semantics in Bellmania.}
\end{figure}


\medskip
At this point, Richard notices that $G^{\tinyqbox1}$ is just a smaller version of
the entire array $G$; he invokes another tactic called {\sf Synth}, which automatically
synthesizes a recursive call A\big[$G^{\tinyqbox1}$\big]
(presented using abstract index ranges as $A^{J_0}$).

Similarly, $G^{\tinyqbox4}$ is also a smaller version of $G$, this time with
indices from $J_1$. {\sf Synth} figures it out automatically as well,
synthesizing a call A\big[$G^{\tinyqbox4}$\big]. The remaining part,
$G^{\tinyqbox2}$, is essentially different, so Richard gives it a new name, ``B'',
which becomes a new synthesis task.%
\footnote{Richard's choice of names is consistent with the literature.}

\begin{algorithm}[b]
\renewcommand\arraystretch{1.3}
\begin{algorithmic}
\Procedure{A{\larger{[}}$G${\larger{]}}}{}\EndProcedure
  \If{$G$ {\it is very small}} {\it run iterative version}
  \Else
  \State A\big[$G^{\tinyqbox1}$\big] \Head{Compute \qbox1}
  \State A\big[$G^{\tinyqbox4}$\big] \Head{Compute \qbox4}
  \State B\big[$G^{\tinyqbox1}, G^{\tinyqbox4}, G^{\tinyqbox2}$\big] \Head{Compute \qbox2}
  \EndIf
\end{algorithmic}
\caption{\label{overview:recursive-A}
   Parenthesis --- Recursive Version}
\end{algorithm}


Applying the same strategy will eventually lead Richard to further
break down and transform the computation of B into multiple recursive
sub-computations, %\rch{B makes eight recursive function calls to update the four quadrants. So it reduces to eight recursive sub-computations, right?}, 
further improving the locality of the resulting algorithm until
a true divide-and-conquer solution is obtained.
Bellmania generates code for
all procedures encountered throughout the development. In this case, three recursive procedures
are generated. The base case of the recursion is when the region
becomes small enough to just run the loop version.\footnote{The optimal  base case size of the region can be found by auto-tuning (taken as an input in the current version of the compiler)}

As is well illustrated by the example, this line of reasoning can get quite complicated for most dynamic programming algorithms, 
and producing a correct divide-and-conquer algorithm for a given dynamic programming problem is considered quite difficult even by the researchers who originally pioneered the technique. 
Fortunately, Bellmania is able to mechanize most of the technical details,
allowing Richard and other algorithm designers to focus on their area of expertise,
try different strategies,
and eventually produce a \emph{certified} implementation of the algorithm.

Overall, it took Richard only 4 steps to construct \Cref{overview:recursive-A},
and a total of 30 steps to construct all three phases of the Parenthesis algorithm,
comprising an implementation that is 46$\times$ faster than a parallel implementation of
\Cref{intro:naive} using a state-of-the-art parallelizing compiler.
The user is greatly assisted by tactics like {\sf Synth}, which carries out the monotonic
and error-prone task of choosing the right parameters for each recursive call; also,
mistakes are identified early in the development thanks to automatic verification,
saving hours of debugging later on.
The resulting code is much easier to maintain, because the artifact is not just
the optimized C++ code, but also the Bellmania specification and the script
that encodes the optimization strategy.

Once a divide-and-conquer algorithm is found, generating an optimal implementation still requires some additional work, such as finding the right point at which to switch to an iterative algorithm to leverage SIMD parallelism as well as low-level tuning and compiler optimization;
these steps are performed by more traditional compiler optimization techniques
as discussed in \Cref{codegen}.


\subsection*{System Design}

\begin{figure}
  \centering
  
  \begin{tikzpicture}[
      h/.style={minimum height=1.75em}, 
      tallh/.style={minimum height=3.1em}, 
      doubleh/.style={minimum height=3.5em+2mm}, 
      fullw/.style={minimum width=5cm}, 
      halfw/.style={minimum width=2.4cm},
      75w/.style={minimum width=1.15cm},
      every node/.style={align=center}]
    \node[draw, h, fullw](tae) {Tactic Application Engine};
    \node[draw, h, fullw, above=2mm of tae](dp) {\fontsize{9.5}{11}\selectfont Divide-and-Conquer Tactic Library};
    \node[draw, tallh, halfw, below=2mm of tae.south west, anchor=north west](verif) {Verifier \\ {\small (SMT)}};
    \node[draw, tallh, halfw, below=2mm of tae.south east, anchor=north east](comp) {Compiler \\ {\small (C++, cilk)}};
    \draw (dp.south) -- (tae.north);
    \draw (tae.south -| verif.north) -- (verif.north);
    \draw (tae.south -| comp.north) -- (comp.north);
    
    \node[draw, doubleh, left=4mm of dp.north west, anchor=north east](ui) {Frontend \\ UI};
    \draw (dp.west) -- (dp.west -| ui.east);
    \draw (tae.west) -- (tae.west -| ui.east);
    
    \node[draw, tallh, right=4mm of comp](exe) {Exec. \\ Code};
    \draw[double] (comp) -- (exe);
  \end{tikzpicture}
\caption{\label{overview:design}
  Overall design of Bellmania.}
\end{figure}

The design of the Bellmania system (\Cref{overview:design}) contains a generic
core --- called the \newterm{Tactic Application Engine} (TAE) ---
on top of which a library of \newterm{tactics} specific to dynamic programming
is built. While it is possible to extend the library, it already contains enough tactics to successfully develop algorithms for a
family of problems, so that the user of the system only needs to apply existing tactics by issuing commands
through a GUI and watching the program evolve. The TAE has a back-end that
verifies conjectures, and is in charge of making sure that
tactic applications represent valid rewritings of the program. Finally, the
programs created this way are transferred to a compilation back-end, where
some automatic static analysis is applied and then executable (C++) code is emitted.

\cbstart\diffnote{made the point clear regarding verification vs. certification, and what is missing}%
The trusted core is small, comprising of:
\begin{enumerate*}[label=(\textit{\alph*})]
 \item a type checker (see \Cref{lang:types}), 
 \item a term substitution procedure (see \Cref{tactics}),
 \item a formula simplifier (see \Cref{automated:simplification}),
 \item the SMT solver, and
 \item the compiler (see \Cref{codegen}).
\end{enumerate*}
In its current version, Bellmania does not emit machine-checkable
proofs; to the extent that SMT solvers can emit such proofs,
these could be adapted to a proof of refinement (in the sense of \cite{POPL15/Delaware}).
The compiler has to be formally verified separately.
These two tasks require considerable engineering effort and are left for the future.
\cbend

\begin{figure}
\vspace{-.5em}% why is there extra space above lstlisting?
\begin{lstlisting}[language=bellmania]
Slice (find (\theta |-> ?)) (? <|J_0*J_0, J_0*J_1, J_1*J_1|>)
Stratify "/" (fixee [A]) [A] \psi
Stratify "/" (fixee [A]) [A] \psi
[A] [B] [C] |-> SynthAuto . ... \psi
\end{lstlisting}
\caption{\label{overview:script}
  Bellmania script used to generate \Cref{overview:recursive-A}.}
\end{figure}

An example for the concrete syntax is shown in \Cref{overview:script}.
A full listing of the scripts for our running examples,
as well as screenshots of the UI, can be found
in \Cref{annex:example:paren}.

\subsection*{Interaction Model}

\begin{figure}[b!]
\begin{tikzpicture}[>=latex,
     blk/.style={draw,rectangle,align=center,inner sep=5pt,minimum height=8mm}]
  \node(spec)[blk]{Specification};
  \node(prog)[blk,below right=8mm of spec]{Program Term};
  \node(deriv)[blk,above=of prog]{Derived\\Definition};
  \node(impl)[blk,above right=8mm of prog]{Implementation};
  \draw[->,rounded corners=5mm] (spec) |- node[below]{\textit{typecheck}} (prog);
  \draw[->] (prog) edge[out=112,in=-105] (deriv);
  \draw[->] (deriv) edge[out=-73,in=70] (prog);
  \draw[->,rounded corners=5mm] (prog) -| node[below]{\textit{compile}} (impl);
  % Loop over (prog)
  \draw[->,rounded corners=3mm] (prog.-10) -| ++(.3,-.6) -| ($(prog.190) + (-.3,0)$) -- (prog.190);
  \node[below=6mm of prog]{\textit{apply tactic}};
\end{tikzpicture}
\caption{\label{overview:flow}
  Interaction workflow in Bellmania.}
\end{figure}

\cbstart\diffnote{added subsection (incl. figure)}%
The intended use pattern for Bellmania is by repeatedly issuing
commands to apply tactics inside a REPL that keeps showing to the
user the resulting \newterm{program term} (\Cref{overview:flow}).
To begin the interaction, the user types in a \newterm{specification} written
in the Bellmania language (as defined in \Cref{lang}), which is typechecked
by the system, producing a term.
This term then becomes the focus of the development, and further transformations
apply to it; the user issues \newterm{apply tactic} commands, one at a time,
using a syntax similar to \Cref{overview:script}.
Tactic applications are also typechecked, as well as the resulting program,
since type information can be refined by including the context (see \Cref{lang:types}).
During the development, sub-computations may be discovered that require
further drilling down (such as ``B'' in the last example).
These result in \newterm{derived} terms that take the focus until they, too,
have been fully transformed.
When the program and all its sub-computations are fully developed and
expressed as a divide-and-conquer algorithm, the user invokes the compiler
back-end that emits C++ code.

In the following sections, we describe the programming language and
formally define the tactics that were used in the example above.
We then show how to formalize the same intuition as we had there,
using this new instrument.
\cbend
