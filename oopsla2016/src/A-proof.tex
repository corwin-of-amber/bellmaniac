\section{Proof of Soundness}
\label{annex:proof}

\renewenvironment{proof}{\noindent{\bf Proof.~}}{}

\begin{theorem}
Let $s=s'$ be an instance of one of the tactics introduced in this section.
let $a_i=b_i$, $i=1..k$, be the proof obligations. If $\semp{a_i}=\semp{b_i}$
for all interpretations of the free variables of $a_i$ and $b_i$, then
$\semp{s}=\semp{s'}$ for all interpretations of the free variables of $s$ and $s'$.
\end{theorem}

\begin{proof}
For the tactics with \textbf{\textit{Obligations}:} tactic, the theorem is trivial.

\medskip
\noindent
{\tt >} For {\sf Stratify}, let $f$, $g$ be partial functions such that
\vspace{-.5em}
\[\renewcommand\arraystretch{1.3}
  \forall \theta,\zeta.\quad \begin{array}{l}f\,(g\,\zeta\,\theta) ~=~ f\,\zeta \quad\land\quad
  g\,(f\,\theta)\,\theta ~=~ h\,\theta
  \end{array}\quad\]
  
Assume that $\zeta = \fix f$ and $\theta = \fix (g\,\zeta)$. 
That is, $f\,\zeta = \zeta$ and $g\,\zeta\,\theta = \theta$.
Then ---
\vspace{-.5em}
\[\renewcommand\arraystretch{1.3}
  \begin{array}{l@{}l}
   h\,\theta & {}= g\,(f\,\theta)\,\theta = g\,(f\,(g\,\zeta\,\theta))\,\theta =
              g\,(f\,\zeta)\,\theta = \theta
  \end{array}\]
  
\noindent
So $\theta = \fix h$. We get $\fix h = \fix \big(g \,(\fix f)\big)$; equivalently,
\[\fix h = (\fix f) \applt \big(\psi\mapsto\fix (g\,\psi)\big)\]

Now instantiate $h$, $f$, and $g$, with $f\applt g$, $f$, and $g'$ from \Cref{tactics:Stratify obligations},
and we obtain the equality in the tactic.

\medskip
\noindent
{\tt >} For {\sf Synth}, ({\it i}) assume $f_i=\fix g$ and
\[h :: \T\to\Y = h :: \Y\to\Y = g :: \Y\to\T\]

Intuitively, $\Y$ ``cuts out'' a region of an array $\theta :: \T$ given
as input to $h$ and $g$. This area is self-contained, in the sense that
only elements in $\Y$ are needed to compute elements in $\Y$, as indicated
by the refined type $\Y\to\Y$.

Notice that from the premise follows $g :: \Y\to\T = g :: \Y\to\Y$. We use the following corollary:

\medskip\noindent
{\bf Corollary.~} Let $f : \T\to\T$; if either $f :: \T\to\Y = f :: \Y\to\Y$ or $f :: \Y\to\T = f :: \Y\to\Y$, 
then $(\fix f)::\Y = \fix (f :: \Y\to\Y)$.

Proof follows later in this appendix.

\medskip
From the corollary, and for the given $h$ and $g$, we learn that $(\fix h)::\Y = \fix(h::\Y\to\Y)$,
and also $(\fix g)::\Y=\fix(g::\Y\to\Y)$. Since $h::\Y\to\Y=g::\Y\to\Y$,
we get $(\fix h)::\Y = (\fix g)::\Y$; now, $\Y$ is a supertype of $\T_i$, so $(\theta::\Y)::\T_i=\theta::\T_i$:
\[\renewcommand\arraystretch{1.3}
  \begin{array}{l}(\fix h)::\T_i = ((\fix h)::\Y)::\T_i = ((\fix g)::\Y)::\T_i = \\
    \qquad = (\fix g)::\T_i=f_i::\T_i
  \end{array}\]

{\it (ii)} Assume $h\,(h\,\theta) :: \T_i = f_i :: \T_i$ holds for any $\theta:\T$,
then in particlar, for $\theta=\fix\,h$, we get $h\,(h\,\fix h) :: \T_i = f_i :: \T_i$.
Since $h\,(h\,\fix h) = \fix h$, we obtain the conjecture $(\fix h) :: \T_i = f_i :: \T_i$.
\qed
\end{proof}

\medskip
Our reliance on the termination of $\fix$ expressions may seem conspicuous, since some of these
expressions are generated automatically by the system. However, a closer look reveals that whenever
such a computation is introduced, the set of the recursive calls it makes is a subset of those made by the existing one.
Therefore, if the original recurrence terminates, so does the new one. In any case, all the recurrences
in our development have a trivial termination argument (the indexes $i$,$j$ change monotonically between calls),
so practically, this should never become a problem.

\bigskip

We now prove the corollary from the proof of {\sf Synth}.

\medskip\noindent
{\bf Corollary.~} Let $f : \T\to\T$; if either $f :: \T\to\Y = f :: \Y\to\Y$ or $f :: \Y\to\T = f :: \Y\to\Y$, 
then $(\fix f)::\Y = \fix (f :: \Y\to\Y)$.

\begin{proof}

 For the first case, assume $\theta=\fix f$,

\[\renewcommand\arraystretch{1.3}
  \begin{array}{l}
   \theta::\Y = (\theta\applt f)::\Y = \theta\applt(f::\T\to\Y) = \\
   \qquad=\theta\applt(f::\Y\to\Y) = (\theta::\Y)\applt(f::\Y\to\Y)
 \end{array}\]

This means that $\theta::\Y = \fix(f::\Y\to\Y)$, as desired.
For the second case, from domain theory we know that $\fix f = f^k\bot$ for some $k\geq 1$.
We prove by induction that $f^k\bot = (f :: \Y\to\Y)^k\bot$.

For $k=1$, \[f\,\bot = f\,(\bot :: \Y) = (f :: \Y\to\T)\,\bot = (f :: \Y\to\Y)\,\bot\]

Assume $f^k\bot = (f :: \Y\to\Y)^k\bot$, then definitely $f^k\bot = f^k\bot :: \Y$.
Therefore, 
\[\renewcommand\arraystretch{1.3}
  \begin{array}{l}
    f^{k+1}\bot = (f^k\bot)\applt f = (f^k\bot::\Y)\applt f = \\
    \qquad = (f^k\bot)\applt(f::\Y\to\T) = \\
    \qquad = ((f :: \Y\to\Y)^k\bot) \applt (f::\Y\to\Y) = \\
    \qquad = (f::\Y\to\Y)^{k+1}\bot
  \end{array}
\]

From this we learn that $\fix f = \fix(f :: \Y\to\Y) = (\fix f) :: \Y$.
\end{proof}
