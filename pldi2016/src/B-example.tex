\section{Example Development}

The following example shows in more detail the first phase of the development
for the Simplified Arbiter test case,
to give the reader a flavor of the technique and the way the system
does reasoning.

In Bellmania language, the Simplified Arbiter example is specified by
a base case $\Psi$ corresponding to {\it Initialize}, and a computation
part corresponding to {\it Compute}.

\begin{equation}
  \renewcommand\arraystretch{1.5}
  \begin{array}{@{}l@{}l@{}l@{}}
    \Psi ~=~ & \fix \Big(\theta\,i\,j\mapsto
	              [0]_{i=j=0} ~\big/~ [\yw{0j0}]_{i=0} ~\big/~ [\xw{0i0}]_{j=0}\Big)
  \end{array}
\end{equation}

\begin{equation}
  \renewcommand\arraystretch{1.2}
  \begin{array}{@{}l@{}l@{}c@{}c@{}l@{}l@{}}
    A^{^{IJ}} ~=~ 
	      & \psi\mapsto \fix \theta\,& i & j & \mapsto\min\,\langle~ & \psi_{ij} \\
	      & & ^{^{(I)}} & ^{^{(J)}} & & \min \vtyped p I \mapsto\theta_{pj}+w_{pij}, \\
	      & & & & & \min \vtyped q J \mapsto\theta_{iq}+w'_{qji} ~\rangle
  \end{array}
  \label{tactics:arbiter phase A}
\end{equation}

Vertical typeset was used to save some horizontal space, but $\vtyped v\T$
should be read as just $v:\T$.

\medskip

After Richard applies {\sf Slice}, he gets the four quadrants $I_0\times J_0$, $I_0\times J_1$, $I_1\times J_0$, $I_1\times J_1$
(\Cref{overview:quadrants}). The system defined unary qualifiers with the axioms:
\[
\begin{array}{c@{\qquad}c}
  \forall i{:}I.~~I_0(i)\lor I_1(i)   &    \forall i_0{:}I_0,~i_1{:}I_1.~~i_0<i_1 \\
  \forall j{:}J.~~J_0(j)\lor J_1(j)   &    \forall j_0{:}J_0,~j_1{:}J_1.~~j_0<j_1 \\
\end{array}
\]

The program is just about to
grow quite large; to make such terms easy to read and refer to, we provide
boxed letters as labels for sub-terms, using them as abbreviations where they
occur in the larger expression.

\makeatletter
\newcommand{\quadrants@normal}[4]{
  \renewcommand\arraystretch{1.5}
   \begin{array}{c|c}
     #1 & #2 \\ \hline
     #3 & #4
   \end{array}}
\newcommand{\quadrants@small}[4]{
  \renewcommand\arraystretch{0.9}
   \begin{array}{@{~}c@{~}|@{~}c@{~}}
     \scriptstyle #1 & \scriptstyle #2 \\ \hline
     \scriptstyle #3 & \scriptstyle #4
   \end{array}}
\newcommand\quadrants{\@ifstar\quadrants@small\quadrants@normal}
\makeatother

In addition, to allude to the reader's intuition, expressions of the form
$a/b/c/d$ will be written as $\quadrants*{a}{b}{c}{d}$ when the slices
represent quadrants.

\makeatletter
\newcommand{\lbox@small}[1]{ {\setlength{\fboxsep}{1pt}\fbox{\small #1}} }
\newcommand{\lbox@tiny}[1]{ {\setlength{\fboxsep}{1pt}\fbox{\tiny #1}} }
\newcommand\lbox{\@ifstar\lbox@tiny\lbox@small}
\makeatother

\begin{tacticbox}{Slice}
  \begin{array}{@{} l  l @{}}
       f ~=~ \theta\,i\,j\mapsto \cdots \\
       X_1 ~=~ \_\times I_0\times J_0 &
       X_2 ~=~ \_\times I_0\times J_1 \\
       X_3 ~=~ \_\times I_1\times J_0 &
       X_4 ~=~ \_\times I_1\times J_1 \\[.5em]
       \cspan2{\mbox{\small ({\it each} ``\_'' {\it is a fresh type variable})}}
  \end{array}
\end{tacticbox}

\begin{equation}
  \renewcommand\arraystretch{1.2}
  \begin{array}{@{}r@{}l@{}c@{}c@{}l@{}l@{}}
    A^{^{IJ}} =~ & \lspan5{\psi\mapsto \fix \quadrants{\lbox A}{\lbox B}{\lbox C}{\lbox D}} \\
	\lbox A ~=~ & \theta\,& i & j & \mapsto\min\,\langle~ & \psi_{ij} \\
	      & & ^{^{(I_0)}} & ^{^{(J_0)}} & & \min \vtyped p I \mapsto\theta_{pj}+w_{pij}, \\
	      & & & & & \min \vtyped q J \mapsto\theta_{iq}+w'_{qji} ~\rangle \\
	\lbox B ~=~ & \theta\,& i & j & \mapsto\min\,\langle~ & \psi_{ij} \\
	      & & ^{^{(I_0)}} & ^{^{(J_1)}} & & \min \vtyped p I \mapsto\theta_{pj}+w_{pij}, \\
	      & & & & & \min \vtyped q J \mapsto\theta_{iq}+w'_{qji} ~\rangle \\
	\lbox C ~=~ & \theta\,& i & j & \mapsto\min\,\langle~ & \psi_{ij} \\
	      & & ^{^{(I_1)}} & ^{^{(J_0)}} & & \min \vtyped p I \mapsto\theta_{pj}+w_{pij}, \\
	      & & & & & \min \vtyped q J \mapsto\theta_{iq}+w'_{qji} ~\rangle \\
	\lbox D ~=~ & \theta\,& i & j & \mapsto\min\,\langle~ & \psi_{ij} \\
	      & & ^{^{(I_1)}} & ^{^{(J_1)}} & & \min \vtyped p I \mapsto\theta_{pj}+w_{pij}, \\
	      & & & & & \min \vtyped q J \mapsto\theta_{iq}+w'_{qji} ~\rangle \\
  \end{array}
  \label{evaluation:A sliced}
\end{equation}

\begin{tacticbox}{Let}
   e[\square] ~=~ \quadrants*{\square}{\lbox*B}{\lbox*C}{\lbox*D} \qquad
   t ~=~ \lbox A
\end{tacticbox}

\begin{equation}
  A^{^{IJ}} =~ \psi\mapsto \fix \left(\lbox A \applt z\mapsto\quadrants{z}{\lbox B}{\lbox C}{\lbox D}\right)
\end{equation}

\begin{tacticbox}{Stratify[with Padding]}
  \begin{array}{@{} l @{} l @{}}
    f ~=~ \quadrants*{\lbox*A}{\dot\psi}{\dot\psi}{\dot\psi}
         & \mbox{\small ({\it recall that } $\dot\psi=\theta\mapsto\psi$)} \\
    g ~=~ z\mapsto\quadrants*{z}{\lbox*{B}}{\lbox*C}{\lbox*D} &
    \qquad\psi=\psi
  \end{array}
\end{tacticbox}

\begin{equation}
  A^{^{IJ}} =~ \psi\mapsto \fix \quadrants{\lbox A}{\dot\psi}{\dot\psi}{\dot\psi} ~\applt~ \psi\mapsto\fix\quadrants{\dot\psi}{\lbox B}{\lbox C}{\lbox D}
  \label{evaluation:stratify A 1}
\end{equation}

Notice that an existing variable $\psi$ is reused, rebinding any occurrences within $\lbox B$, $\lbox C$, $\lbox D$.
This effect is useful, as it limits the context of the expression: the inner $\psi$ shadows the outer $\psi$,
meaning $\lbox B$, $\lbox C$, $\lbox D$ do not need to access the data that was input to $\lbox A$, only its
output.

\medskip
The sequence Let, Stratify[with Padding] is now applied in the same manner to $\lbox B$
and $\lbox C$ (see \Cref{overview:slice-stratify-synth}). We do not list the applications as they are analogous to the previous ones.

\begin{equation}
  \renewcommand\arraystretch{1.5}
  \begin{array}{l@{}l}
    A^{^{IJ}} =~ \psi\mapsto{} & \fix \quadrants{\lbox A}{\dot\psi}{\dot\psi}{\dot\psi} ~\applt~ 
                 \psi\mapsto\fix\quadrants{\dot\psi}{\lbox B}{\dot\psi}{\dot\psi} ~\applt \\
               & \psi\mapsto\fix\quadrants{\dot\psi}{\dot\psi}{\lbox C}{\dot\psi} ~\applt~
                 \psi\mapsto\fix\quadrants{\dot\psi}{\dot\psi}{\dot\psi}{\lbox D}
  \end{array}
\end{equation}

\begin{tacticbox}{Synth}
	\begin{array}{@{}l@{}c@{}c@{}l@{}l}
       \lspan5{h_1= \lbox A} \\
       \lspan5{h_{2,3,4}=\dot\psi} \\
	   f_1 = \theta\,& i & j & \mapsto\min\,\langle~ & \psi_{ij} \\
	      & ^{^{(I_0)}} & ^{^{(J_0)}} & & \min \vtyped p {I_0} \mapsto\theta_{pj}+w_{pij}, \\
	      & & & & \min \vtyped q {J_0} \mapsto\theta_{iq}+w'_{qji} ~\rangle \\
	   \lspan5{f_{2,3,4} = \dot\psi}
   \end{array}
\end{tacticbox}

\begin{equation}
  \renewcommand\arraystretch{1.5}
  \begin{array}{l@{}l}
    A^{^{IJ}} =~ \psi\mapsto{} & \quadrants{A^{^{I_0J_0}}_\psi\!\!\!}{\psi}{\psi}{\psi} ~\applt~ 
                 \psi\mapsto\fix\quadrants{\dot\psi}{\lbox B}{\dot\psi}{\dot\psi} ~\applt \\
               & \psi\mapsto\fix\quadrants{\dot\psi}{\dot\psi}{\lbox C}{\dot\psi} ~\applt~
                 \psi\mapsto\fix\quadrants{\dot\psi}{\dot\psi}{\dot\psi}{\lbox D}
  \end{array}
  \label{fix A}
\end{equation}

We note that $\fix f_1=A^{^{I_0J_0}}_\psi$ are identical (up to $\beta$-reduction),
which is the whole reason $f_1$ was chosen. Also, we took the liberty
to simplify $\fix\dot\psi$ into $\psi$ --- although this is not necessary --- just to display
a shorter term.

\medskip
The next few tactics will focus on the subterm $\lbox B$ from \eqref{evaluation:A sliced}.

\begin{equation}
  \renewcommand\arraystretch{1.2}
  \begin{array}{@{}r@{}l@{}c@{}c@{}l@{}l@{}}
	\lbox B ~=~ & \theta\,& i & j & \mapsto\min\,\langle~ & \psi_{ij} \\
	      & & ^{^{(I_0)}} & ^{^{(J_1)}} & & \min \vtyped p I \mapsto\theta_{pj}+w_{pij}, \\
	      & & & & & \min \vtyped q J \mapsto\theta_{iq}+w'_{qji} ~\rangle
  \end{array}
\end{equation}

\begin{tacticbox}{Slice}
  \begin{array}{@{} l @{}}
    f = \vtyped q J \mapsto \theta_{iq}+w'_{qji} \\
    X_1 = J_0\to\_ \qquad X_2 = J_1\to\_
  \end{array}
\end{tacticbox}

\begin{equation}
  \renewcommand\arraystretch{1.2}
  \begin{array}{@{}r@{}l@{}c@{}c@{}l@{}l@{}l@{}}
	\lbox B ~=~ & \theta\,& i & j & \mapsto\min\,\big\langle~ & \psi_{ij} \\
	      & & ^{^{(I_0)}} & ^{^{(J_1)}} & & \lspan2{\min \vtyped p I \mapsto\theta_{pj}+w_{pij},} \\
	      & & & & & \min \big( & (\vtyped q {J_0} \mapsto\theta_{iq}+w'_{qji}) ~\big/~ \\
	      & & & & & & (\vtyped q {J_1} \mapsto\theta_{iq}+w'_{qji})\big)  ~\big\rangle
  \end{array}
  \label{evaluation:horiz sliced}
\end{equation}

For the intuition behind this, see the top-right part of \Cref{evaluation:slicing strategy}.
The colors represent cell ranges that will be read by different sub-routines (presumably running on different cores).
The range of $q$ is split into the part that lies within \qbox1 ($q\in J_0$) and the one that
lies within \lbox2 ($q\in J_1$). The same reasoning is applied to the other quadrants.

\begin{tacticbox}{Distributivity}
  \begin{array}{@{} l @{}}
    e[\square] = \min\square \\
    t_1 = \min \vtyped q {J_0} \mapsto \theta_{iq}+w'_{qji} \\
    t_2 = \min \vtyped q {J_1} \mapsto \theta_{iq}+w'_{qji} \\
  \end{array}
\end{tacticbox}

\begin{tacticbox}{Associativity}
  \begin{array}{@{} l @{} l @{}}
    \lspan2{\reduce = \min} \\
    \overline x_1 ={} & \psi_{ij} \\
    \overline x_2 ={} & \min \vtyped p I \mapsto\theta_{pj}+w_{pij} \\
    \overline x_3 ={} & \min \vtyped q {J_0} \mapsto \theta_{iq}+w'_{qji} ~, \\
                      & \min \vtyped q {J_1} \mapsto \theta_{iq}+w'_{qji}
  \end{array}
\end{tacticbox}

\begin{equation}
  \renewcommand\arraystretch{1.2}
  \begin{array}{@{}r@{}l@{}c@{}c@{}l@{}l@{}}
	\lbox B ~=~ & \theta\,& i & j & \mapsto\min\,\big\langle~ & \psi_{ij} \\
	      & & ^{^{(I_0)}} & ^{^{(J_1)}} & & \min \vtyped p I \mapsto\theta_{pj}+w_{pij}, \\
	      & & & & & \min \vtyped q {J_0} \mapsto\theta_{iq}+w'_{qji}, \\
	      & & & & & \min \vtyped q {J_1} \mapsto\theta_{iq}+w'_{qji} ~\big\rangle
  \end{array}
\end{equation}

\begin{tacticbox}{Let[$\reduce$]}
  \begin{array}{@{} r @{} l @{}}
    e[\square] ={} & \quadrants{\dot\psi}{\theta\,i\,j\mapsto\square}{\dot\psi}{\dot\psi} \\
    \overline a ={} & \psi_{ij}, ~\min \vtyped q{J_0}\mapsto \theta_{iq}+w'_{qji} \\
    \overline b ={} & \min \vtyped p I \mapsto\theta_{pj}+w_{pij}, \\
                    & \min \vtyped q {J_1} \mapsto\theta_{iq}+w'_{qji}
  \end{array}
\end{tacticbox}

\begin{equation}
  \renewcommand\arraystretch{1.2}
  \begin{array}{r @{} l @{} c @{} c @{} l @{} l}
    \lspan6{
    \quadrants{\dot\psi}{\lbox B}{\dot\psi}{\dot\psi} =
      \fix\left(\lbox E \applt z\mapsto\quadrants{\dot\psi}{\lbox F}{\dot\psi}{\dot\psi}\right) 
    } \\[1.5em]
    ~\lbox E ={} &
      \theta & i & j & \mapsto\min\langle & \psi_{ij}, \\
             & & ^{^{(I_0)}} & ^{^{(J_1)}} &
                                          & \min \vtyped q{J_0}\mapsto\theta_{iq}+w'_{qji} \rangle \\
    ~\lbox F ={} &
      \theta & i & j & \mapsto\min\langle & z_{\theta ij}, \\
             & & ^{^{(I_0)}} & ^{^{(J_1)}} &
                                         & \min \vtyped p I \mapsto\theta_{pj}+w_{pij}, \\
             & & & &                     & \min \vtyped q {J_1} \mapsto\theta_{iq}+w'_{qji}\rangle
  \end{array}
\end{equation}


\begin{figure}
\begin{tikzpicture}[x=4mm,y=4mm, node distance=1cm,
    slicer/.style={ultra thick},
    cell/.style={thick}, dot/.style={fill=BrickRed},
    block C/.style={fill=OliveGreen, fill opacity=0.2},
    block B/.style={fill=Orange, fill opacity=0.2},
    block A/.style={fill=Purple, fill opacity=0.2},
    label position=right,label distance=1pt,every label/.style={inner sep=0}]
  \draw[help lines] (0,0) grid[step=1] (8,8);
  \draw[slicer] (4,0) -- (4,8) (0,4) -- (8,4);
  \fill[block A] (0,5) rectangle (2,6) rectangle (3,8);
  \draw[cell] (2,5) rectangle +(1,1); \path[dot] (2.5,5.5) circle (1.5pt);
  \path (2,8) node[above] {$q\in J_0$} (6,8) node[above] {$q\in J_1$};
  \path (0,0) -- node[midway,above,sloped] {$p\in I_1$} (0,4)
              -- node[midway,above,sloped] {$p\in I_0$} (0,8);

  \tikzset{xshift=4cm}

  \draw[help lines] (0,0) grid[step=1] (8,8);
  \draw[slicer] (4,0) -- +(0,8) (0,4) -- +(8,0);
  \fill[block B] (0,5) rectangle (4,6);
  \fill[block A] (4,5) rectangle (6,6) rectangle (7,8);
  \draw[cell] (6,5) rectangle +(1,1); \path[dot] (6.5,5.5) circle (1.5pt);
  \path (2,8) node[above] {$q\in J_0$} (6,8) node[above] {$q\in J_1$};
  \path (0,0) -- node[midway,above,sloped] {$p\in I_1$} (0,4)
              -- node[midway,above,sloped] {$p\in I_0$} (0,8);
              
  \tikzset{yshift=-4cm,xshift=-4cm}
  
  \draw[help lines] (0,0) grid[step=1] (8,8);
  \draw[slicer] (4,0) -- +(0,8) (0,4) -- +(8,0);
  \fill[block C] (2,4) rectangle (3,8);
  \fill[block A] (0,1) rectangle (2,2) rectangle (3,4);
  \draw[cell] (2,1) rectangle +(1,1); \path[dot] (2.5,1.5) circle (1.5pt);
  \path (2,8) node[above] {$q\in J_0$} (6,8) node[above] {$q\in J_1$};
  \path (0,0) -- node[midway,above,sloped] {$p\in I_1$} (0,4)
              -- node[midway,above,sloped] {$p\in I_0$} (0,8);

  \tikzset{xshift=4cm}
  
  \draw[help lines] (0,0) grid[step=1] (8,8);
  \draw[slicer] (4,0) -- +(0,8) (0,4) -- +(8,0);
  \fill[block C] (6,4) rectangle (7,8);
  \fill[block B] (0,1) rectangle (4,2);
  \fill[block A] (4,1) rectangle (6,2) rectangle (7,4);
  \draw[cell] (6,1) rectangle +(1,1); \path[dot] (6.5,1.5) circle (1.5pt);
  \path (2,8) node[above] {$q\in J_0$} (6,8) node[above] {$q\in J_1$};
  \path (0,0) -- node[midway,above,sloped] {$p\in I_1$} (0,4)
              -- node[midway,above,sloped] {$p\in I_0$} (0,8);
  
  \tikzset{yshift=-.7cm,xshift=-2cm}
  
  \node(A)[rectangle,label=$A$,block A] {};
  \node(B)[right=of A,rectangle,label=$B$,block B] {};
  \node(C)[right=of B,rectangle,label=$C$,block C] {};
  
\end{tikzpicture}

\medskip
\caption{\label{evaluation:slicing strategy}
  The strategy for applications of {\sf Slice} in the case study.}
\end{figure}

\begin{tacticbox}{Stratify[with Padding]}
  \begin{array}{@{} l @{}}
    f ~=~ \quadrants*{\dot\psi}{\lbox*E}{\dot\psi}{\dot\psi} \\
    g ~=~ z\mapsto\quadrants*{\dot\psi}{\lbox*F}{\dot\psi}{\dot\psi}
    \qquad\psi=\psi
  \end{array}
\end{tacticbox}

\begin{equation}
  \renewcommand\arraystretch{1.2}
  \begin{array}{r @{} l @{} c @{} c @{} l @{} l}
    \lspan6{
    \fix\quadrants{\dot\psi}{\lbox B}{\dot\psi}{\dot\psi} =
      \fix\quadrants{\dot\psi}{\lbox E}{\dot\psi}{\dot\psi} \applt
      \psi\mapsto\fix\quadrants{\dot\psi}{\lbox F}{\dot\psi}{\dot\psi}
    } \\[1.5em]
    ~\lbox E ={} &
      \theta & i & j & \mapsto\min\langle & \psi_{ij}, \\
             & & ^{^{(I_0)}} & ^{^{(J_1)}} &
                                          & \min \vtyped q{J_0}\mapsto\theta_{iq}+w'_{qji} \rangle \\
    ~\lbox F ={} &
      \theta & i & j & \mapsto\min\langle & \psi_{ij}, \\
             & & ^{^{(I_0)}} & ^{^{(J_1)}} &
                                          & \min \vtyped p I \mapsto\theta_{pj}+w_{pij}, \\
             & & & &                      & \min \vtyped q {J_1} \mapsto\theta_{iq}+w'_{qji}\rangle
  \end{array}
\end{equation}

\noindent
Define
\begin{equation}
  \renewcommand\arraystretch{1.2}
  \begin{array}{@{}l @{} l @{\!} c @{} c @{} l @{} l@{}}
  B^{^{IJ_0J_1}} =~ \big( & \psi\mapsto \\
      & \fix
      \theta & i & j & \mapsto\min\langle & \psi_{ij}, \\
           & & ^{^{(I)}} & ^{^{(J)}} &
                                          & \min \vtyped q{J_0}\mapsto\theta_{iq}+w'_{qji} \rangle\big) \\
      & \lspan5{:: \big((I\times J_0)\to\R\big) \to \big((I\times J_1)\to\R\big)}
  \end{array}
\end{equation}

\begin{tacticbox}{Synth}
  \begin{array}{@{} l @{} c @{} c @{} l @{} l @{}}
    \lspan5{h_2 = \lbox E} \\
    \lspan5{h_{1,3,4} = \dot\psi} \\
    f_2 = 
      \theta & i & j & \mapsto\min\langle & \psi_{ij}, \\
             & ^{^{(I_0)}} & ^{^{(J_1)}} &
                                          & \min \vtyped q {J_0} \mapsto\theta_{iq}+w'_{qji}\rangle \\
    \lspan5{f_{1,3,4} = \dot\psi}
  \end{array}
\end{tacticbox}

\begin{tacticbox}{Synth}
  \begin{array}{@{} l @{} c @{} c @{} l @{} l @{}}
    \lspan5{h_2 = \lbox F} \\
    \lspan5{h_{1,3,4} = \dot\psi} \\
    f_2 = 
      \theta & i & j & \mapsto\min\langle & \psi_{ij}, \\
             & ^{^{(I_0)}} & ^{^{(J_1)}} &
                                          & \min \vtyped p {I_0} \mapsto\theta_{pj}+w_{pij}, \\
             & & &                        & \min \vtyped q {J_1} \mapsto\theta_{iq}+w'_{qji}\rangle \\
    \lspan5{f_{1,3,4} = \dot\psi}
  \end{array}
\end{tacticbox}

\begin{equation}
  \fix\quadrants{\dot\psi}{\lbox B}{\dot\psi}{\dot\psi} ~=~
    \quadrants{\psi}{B^{^{I_0J_0J_1}}_\psi\!\!\!\!}{\psi}{\psi} ~\applt~
    \psi\mapsto\quadrants{\psi}{A^{^{I_0J_1}}_\psi\!\!\!}{\psi}{\psi}
  \label{fix B}
\end{equation}

\medskip\noindent
In a similar manner, we will obtain the following:

\begin{equation}
  \fix\quadrants{\dot\psi}{\dot\psi}{\lbox C}{\dot\psi} ~=~
    \quadrants{\psi}{\psi}{C^{^{I_0I_1J_0}}_\psi\!\!\!\!}{\psi} ~\applt~
    \psi\mapsto\quadrants{\psi}{\psi}{A^{^{I_1J_0}}_\psi\!\!\!}{\psi}
  \label{fix C}
\end{equation}

\begin{equation}
  \renewcommand\arraystretch{1.2}
  \begin{array}{@{}l @{} l @{\!} c @{} c @{} l @{} l@{}}
  C^{^{I_0I_1J}} =~ \big( & \psi\mapsto \\
      & \fix
      \theta & i & j & \mapsto\min\langle & \psi_{ij}, \\
           & & ^{^{(I)}} & ^{^{(J)}} &
                                          & \min \vtyped p {I_0} \mapsto\theta_{pj}+w_{pij} \rangle\big) \\
      & \lspan5{:: \big((I_0\times J)\to\R\big) \to \big((I_1\times J)\to\R\big)}
  \end{array}
\end{equation}

\medskip\noindent
And ---

\begin{equation}
  \begin{array}{@{} l @{} l @{}}
    \fix\quadrants{\dot\psi}{\dot\psi}{\dot\psi}{\lbox D} ~=~ &
      \quadrants{\psi}{\psi}{\psi}{B^{^{I_1J_0J_1}}_\psi\!\!\!\!} ~\applt~
      \psi\mapsto\quadrants{\psi}{\psi}{\psi}{C^{^{I_0I_1J_1}}_\psi\!\!\!} \\
    &
       ~\applt~ \psi\mapsto\quadrants{\psi}{\psi}{\psi}{A^{^{I_1J_1}}_\psi\!\!\!}
  \end{array}
  \label{fix D}
\end{equation}

This gives the stratified version as shown in \Cref{evaluation:arbiter stratify A chain}.
The read and write regions are already encoded in the types of $A$, $B$, $C$ in 
\eqref{fix A}, \eqref{fix B}, \eqref{fix C}, and \eqref{fix D}.

\begin{figure*}
\centering
\begin{tikzpicture}[>=latex,x=7mm,y=7mm,
    every path/.style={step=1},
    every node/.style={inner sep=.5pt},
    below edge/.style={below=1mm},
    above edge/.style={above=1mm},
    block/.style={rectangle,draw,thick,fill=blue, fill opacity=0.15, inner sep=0}]
    
  \def\dx{2.1cm}
  \def\dy{1cm}
  \def\w{3mm}
    
  \draw (0,0) grid (2,2);
  \node(1) at (.5,1.5) {1};   \node(2) at (1.5,1.5) {2};
  \node(3) at (.5,.5) {3};    \node(4) at (1.5,.5) {4};
  \node(s)[block,fit={(0,1) (1,2)}] {};

  \node[inner sep=0] at (2.5,1) {\includegraphics[width=\w]{img/arrow}};
  
  \tikzset{xshift=\dx}
  
  \draw (0,0) grid (2,2);
  \node(1) at (.5,1.5) {\,1$'$};   \node(2) at (1.5,1.5) {2};
  \node(3) at (.5,.5) {3};    \node(4) at (1.5,.5) {4};
  \draw[->] (s.60) to[out=30] node[above edge] {$A$} (1);
  \node(s)[block,fit={(.05,1) (2.05,2)}] {};
  \node(t)[block,fit={(0,2.05) (1,0)}] {};

  \node[inner sep=0] at (2.5,1) {\includegraphics[width=\w]{img/arrow}};
  
  \tikzset{xshift=\dx, yshift=\dy}
 
  \draw (0,0) grid (2,2);
  \node(1) at (.5,1.5) {\,1$'$};   \node(2) at (1.5,1.5) {\,2$'$};
  \node(3) at (.5,.5) {3};    \node(4) at (1.5,.5) {4};
  \draw[->] (s.60) to[out=80,looseness=1.2] node[above edge] {$B$} (2.120);
  \node(s)[block,fit={(1,1) (2,2)}] {};

  \tikzset{yshift=-2*\dy}
 
  \draw (0,0) grid (2,2);
  \node(1) at (.5,1.5) {\,1$'$};   \node(2) at (1.5,1.5) {2};
  \node(3) at (.5,.5) {\,3$'$};    \node(4) at (1.5,.5) {4};
  \draw[->] (t.-80) to[out=-60,in=-140] node[below edge] {$C$} (3.-120);
  \node(t)[block,fit={(0,0) (1,1)}] {};

  \tikzset{yshift=\dy} % back to middle

  \node[inner sep=0] at (2.5,1) {\includegraphics[width=\w]{img/arrow}};
  
  \tikzset{xshift=\dx}
  
  \draw (0,0) grid (2,2);
  \node(1) at (.5,1.5) {\,1$'$};   \node(2) at (1.5,1.5) {\,\,2$''$};
  \node(3) at (.5,.5) {\,\,3$''$};    \node(4) at (1.5,.5) {4};
  \draw[->] (s) to[out=40,in=120] node[above edge,xshift=1mm] {$A$}  (2);
  \draw[->] (t) to[out=-60,in=-120] node[below edge] {$A$}  (3);
  \node(s)[block,fit={(1,0) (2,2)}] {};

  \node[inner sep=0] at (2.5,1) {\includegraphics[width=\w]{img/arrow}};

  \tikzset{xshift=\dx}

  \draw (0,0) grid (2,2);
  \node(1) at (.5,1.5) {\,1$'$};   \node(2) at (1.5,1.5) {\,\,2$''$};
  \node(3) at (.5,.5) {\,\,3$''$};    \node(4) at (1.5,.5) {\,4$'$};
  \draw[->] (s.-80) to[out=-40,in=-120] node[below edge] {$C$}  (4);
  \node(s)[block,fit={(0,0) (2,1)}] {};

  \node[inner sep=0] at (2.5,1) {\includegraphics[width=\w]{img/arrow}};

  \tikzset{xshift=\dx}

  \draw (0,0) grid (2,2);
  \node(1) at (.5,1.5) {\,1$'$};   \node(2) at (1.5,1.5) {\,\,2$''$};
  \node(3) at (.5,.5) {\,\,3$''$};    \node(4) at (1.5,.5) {\,\,4$''$};
  \draw[->] (s.-40) to[out=-40,in=-120] node[below edge] {$B$} (4);
  \node(s)[block,fit={(1,0) (2,1)}] {};

  \node[inner sep=0] at (2.5,1) {\includegraphics[width=\w]{img/arrow}};

  \tikzset{xshift=\dx}

  \draw (0,0) grid (2,2);
  \node(1) at (.5,1.5) {\,1$'$};   \node(2) at (1.5,1.5) {\,\,2$''$};
  \node(3) at (.5,.5) {\,\,3$''$};    \node(4) at (1.5,.5) {\,\,\,4$'''$};
  \draw[->] (s.-80) to[out=-40,in=-120] node[below edge] {$A$} (4);

\end{tikzpicture}
\caption{\label{evaluation:arbiter stratify A chain}
  Fully divide-and-conquered version of $A^{IJ}$ in the example development.}
\end{figure*}


\medskip
\hrule
\bigskip
