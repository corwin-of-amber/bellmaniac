\section{Related Work}
\label{related}

\begin{itemize}
\color{Gray}
\item Manohar Jonnalagedda: \cite{OOPSLA14/Jonnalagedda} \footnote{http://dl.acm.org/citation.cfm?doid=2660193.2660241}
\item Refinement Calculator \cite{TPHOLs96/Butler} \footnote{http://citeseerx.ist.psu.edu/viewdoc/summary?doi=10.1.1.28.4446}
\item Doug Smith: Divide-and-conquer\cite{AI85/Smith} \footnote{http://www.sciencedirect.com/science/article/pii/0004370285900839}
\end{itemize}

Our ``$\big/$'' operator can be compared to the separating disjunction ``$\ast$'' of Separation Logic~\cite{LICS02/Reynolds},
used to frame parts of the dynamic heap (which can be thought of as one large array),
in particular while checking that a program only accesses the parts allocated to it in its precondition.
While $\ast$ has the semantics of an existentially quantified predicate, Bellmania uses type qualifiers
to explicitly specify a formula defining each part. In this sense, it is more closely related to
Region Logic~\cite{ECOOP08/Banerjee}. These formulas make encoding in first-order logic straightforward,
and the use of Liquid Types allows for any number of dimensions and for decidable checking of domain inclusion
and disjointness.
