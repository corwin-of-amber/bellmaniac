\section{Tactics}

We now define the method by which that our framework transforms program terms, by means of \newterm{tactics}.
A tactic is a scheme of equalities that can be used for rewriting.
When applied to a program term, any occurrence of the left-hand side is replaced by the right-hand side.
A valid application of a tactic is an instance of the scheme that is well-typed and logically valid
(that is, the two sides have the same interpretation in any structure that interprets the free
variables occurring in the equality).

The application of tactics yields a sequence of program terms, each of which is checked to
be equivalent to the previous one. We refer to this sequence by the name \newterm{development}.

We associate with each tactic some \newterm{proof obligations}, listed after the word \textbf{\textit{Obligations}}
in the following paragraph.
When applying a tactic instance, these obligations are also instantiated and given to an automated prover. 
If verified successfully, they entail the validity of the instance. 
Clearly the tactic itself can be used as its proof obligation, if it is easy enough to prove automatically; 
in such cases we write ``\textbf{\textit{Obligations}:} tactic.''

The following are the major tactics exposed by our framework. 
More tactic definitions are given in the appendix.

\newcommand\Obligations{\medskip\noindent\textbf{\textit{Obligations}:} }
\newcommand\reduce{\operatorname{reduce}}
\newcommand\listConcat{{\scriptstyle \,++\,}}

\theoremstyle{definition}
\newtheorem{tactic}{Tactic}

\newcommand\tacticdef[1]{\subsection*{#1}}

\tacticdef{Slice} \label{tactics:Slice}
\[f ~=~ f\big|_{X_1} ~\Big/~ f\big|_{X_2} ~\Big/ ~\cdots~ \Big/~ f\big|_{X_r}\] 

This tactic partitions a mapping into sub-regions. Each $X_i$ may be a $\times$-expression
according to the arity of $f$.

\Obligations tactic.

Informally, the recombination expression is equal to $f$
when $X_{1..r}$ ``cover'' all the defined points of $f$ (also known as the \newterm{support} of $f$).

\tacticdef{Shrink} \label{tactics:Shrink}
\[f ~=~ f :: \T\] 

Used to extra-specify the type of a sub-term.

\Obligations tactic.

For arrow-typed terms, this essentially requires to prove that $f$ is undefined anyway for
argument values outside the domain of $\T$, and that the defined values are in the range of $\T$.

\tacticdef{Stratify} \label{tactics:Stratify}
\[\fix (f\applt g) ~=~ \fix f ~\applt~ \psi\mapsto \fix (\dot\psi\applt g)\]
%
where $\dot\psi$ abbreviates $\theta\mapsto\psi$, with fresh variable $\theta$.

$\psi$ may be fresh, or it may reuse a variable already occurring in $g$, rebinding those occurrences.
The example of this section will illustrate why this is useful.

\Obligations Let $h=f\applt g$ and $g'=\psi\mapsto\dot\psi\applt g$. Let $\theta,\zeta$ be
fresh variables.
\begin{equation}
\renewcommand\arraystretch{1.5}
\begin{array}{l}
f\,(g'\,\zeta\,\theta) ~=~ f\,\zeta \\
g'\,(f\,\theta)\,\theta ~=~ h\,\theta
\end{array}
\label{tactics:Stratify obligations}
\end{equation}

Although the proof is not hard, we defer it to a later theorem.

\tacticdef{Synth} \label{tactics:Synth}
\[\fix\big(h_1 ~\big/~ \cdots ~\big/~ h_r\big) ~=~ 
  (\fix f_1) :: \T_1 ~\big/~ \cdots ~\big/~ (\fix f_r) :: \T_r\]

This tactic is used to generate recursive calls to sub-programs. For $i=1..r$, $f_i$
is typically either $h_i$ or a sub-term occurring earlier in the development.

\Obligations Let $h=h_1/\cdots/h_r$, let $\overline\theta\!=\!\theta_{1..r}$ be $r$ fresh variables, and let
$f = \theta_{1..r} \mapsto (f_1\,\theta_1)::\T_1/\cdots/(f_r\,\theta_r)::\T_r$.
\begin{itemize}
  \item $\T_{1..r}$ are disjoint mappings.
  \item $h\,(f\,\overline\theta) = f\,\overline\theta$.
\end{itemize}

\subsection{Example \hrulefill}

\newenvironment{tacticbox}[1]{\begin{center}\begin{tabular}{|@{~~~~}l@{~~~~}|}\hline\rule{0pt}{2.3ex}\underline{#1}\\[.4em]$}{$\\[-1em] \\[.3ex] \hline\end{tabular}\end{center}}

For simplicity of the example, we assume that the input to the Simplified Arbiter problem
satisfies triangle inequalities:
%
\begin{equation}
w_{pij} \leq w_{pkj} + w_{kij} \qquad w'_{qji} \leq w'_{qki} + w'_{kji}
\label{equ:triangle}
\end{equation}
%
for all (appropriately typed) $p<k<i$, ~$q<k<j$.

\medskip
Starting from the specification in \eqref{lang:arbiter spec}, we apply Synth to turn
$\fix (\theta\,i\,j\mapsto\square/\min\langle\cdots\rangle)$ into $\fix\theta\,i\,j\mapsto\min\langle\square,\cdots\rangle$.
While not absolutely necessary, we will see that it makes some expressions easier to handle later
on.
Distributivity and Associativity (refer to \Cref{more-tactics}) are used to obtain the tactic parameter $f_1$;
the proof is deferred until Synth is applied so that the prover can use the extra context.

\begin{tacticbox}{Synth}
  \begin{array}{@{} l @{} l @{} l @{}}
       h_1=\theta\,i\,j\mapsto{}
	      & \lspan2{0|_{i=0\land j=0} ~\big/~ w'_{0j0}|_{i=0} ~\big/~ w_{0i0}|_{j=0} ~\big/~} \\
	      & \min\,\langle~ & \min p\mapsto\theta_{pj}+w_{pij}, \\
	      & & \min q\mapsto\theta_{iq}+w'_{qji} ~\rangle \\
       f_1=\theta\,i\,j\mapsto{}
	      & \min\,\langle~ & 0|_{i=0\land j=0} ~\big/~ w'_{0j0}|_{i=0} ~\big/~ w_{0i0}|_{j=0}, \\
	      & & \min p\mapsto\theta_{pj}+w_{pij}, \\
	      & & \min q\mapsto\theta_{iq}+w'_{qji} ~\rangle
  \end{array}
\end{tacticbox}

\begin{equation}
  \renewcommand\arraystretch{1.5}
  \begin{array}{@{}l@{}l@{}l@{}}
    G ~=~ \fix \theta\,i\,j\mapsto{}
	      & \min\,\langle~ & 0|_{i=0\land j=0} ~\big/~ w'_{0j0}|_{i=0} ~\big/~ w_{0i0}|_{j=0}, \\
	      & & \min p\mapsto\theta_{pj}+w_{pij}, \\
	      & & \min q\mapsto\theta_{iq}+w'_{qji} ~\rangle
  \end{array}
\end{equation}

We then apply Let Insertion, followed by Stratify, to separate the base case and obtain a more general 
form, similar to $\Ggen$ of \eqref{intro:Ggen}.

\begin{tacticbox}{Let}
  \begin{array}{l@{}l}
   e[\square] ~=~ \theta\,i\,j\mapsto\min\langle~ & \square, \\
      & \min p\mapsto\theta_{pj}+w_{pij}, \\
      & \min q\mapsto\theta_{iq}+w'_{qji} ~\rangle \\
   \lspan2{~~t ~=~
      0|_{i=0\land j=0} ~\big/~ w'_{0j0}|_{i=0} ~\big/~ w_{0i0}|_{j=0}}
  \end{array}
\end{tacticbox}

\begin{equation}
  \renewcommand\arraystretch{1.5}
  \begin{array}{@{}l@{}l@{}l@{}l@{}}
    G ~=~ & \fix \big(\,& \lspan2{(\theta\,i\,j\mapsto 0|_{i=0\land j=0} ~\big/~ w'_{0j0}|_{i=0} ~\big/~ w_{0i0}|_{j=0})\applt} \\
	      & & z\,\theta\,i\,j \mapsto \min\,\langle~& z_{\theta ij}, \\
	      & & & \min p\mapsto\theta_{pj}+w_{pij}, \\
	      & & & \min q\mapsto\theta_{iq}+w'_{qji} ~\rangle\,\big)
  \end{array}
\end{equation}

\begin{tacticbox}{Stratify}
  \begin{array}{l}
       f ~=~ \theta\,i\,j\mapsto 0|_{i=0\land j=0} ~\big/~ w'_{0j0}|_{i=0} ~\big/~ w_{0i0}|_{j=0} \\
       g ~=~ z\,\theta\,i\,j\mapsto \min\langle z_{\theta ij},\cdots\rangle
  \end{array}
\end{tacticbox}

\begin{equation}
  \renewcommand\arraystretch{1.5}
  \begin{array}{@{}l@{}l@{}l@{}}
    G ~=~ & \lspan2{\big(\fix \theta\,i\,j\mapsto
	              0|_{i=0\land j=0} ~\big/~ w'_{0j0}|_{i=0} ~\big/~ w_{0i0}|_{j=0}\big)\applt} \\
	      & \psi\mapsto \fix \theta\,i\,j\mapsto\min\,\langle~ & \psi_{ij} \\
	      & & \min p\mapsto\theta_{pj}+w_{pij}, \\
	      & & \min q\mapsto\theta_{iq}+w'_{qji} ~\rangle
  \end{array}
\end{equation}

Setting the base case aside, let $A^{IJ}$ be the second term,
where the superscript parameterizes the types of $i$, $j$ and $p$, $q$.

\newcommand\vtyped[2]{\underset{\scriptscriptstyle ( #2 )}{ #1 }}

\begin{equation}
  \renewcommand\arraystretch{1.2}
  \begin{array}{@{}l@{}l@{}c@{}c@{}l@{}l@{}}
    A^{^{IJ}} ~=~ 
	      & \psi\mapsto \fix \theta\,& i & j & \mapsto\min\,\langle~ & \psi_{ij} \\
	      & & ^{^{(I)}} & ^{^{(J)}} & & \min \vtyped p I \mapsto\theta_{pj}+w_{pij}, \\
	      & & & & & \min \vtyped q J \mapsto\theta_{iq}+w'_{qji} ~\rangle
  \end{array}
\end{equation}

Vertical typeset was used to save some horizontal space, but $\vtyped v\T$
should be read as just $v:\T$.

\bigskip
Next, we apply Slice to get the four quadrants. $I_0$, $I_1$, $J_0$, $J_1$
(\Cref{intro:quadrants}) are defined as unary qualifiers with the axioms:
\[
\begin{array}{c@{\qquad}c}
  \forall i{:}I.~~I_0(i)\lor I_1(i)   &    \forall i_0{:}I_0,~i_1{:}I_1.~~i_0<i_1 \\
  \forall j{:}J.~~J_0(j)\lor J_1(j)   &    \forall j_0{:}J_0,~j_1{:}J_1.~~j_0<j_1 \\
\end{array}
\]

As a result, the term is
going to grow quite large; to make such terms easy to read and refer to, we provide
boxed letters as labels for sub-terms, using them as abbreviations where they
occur in the larger expression.

\makeatletter
\newcommand{\quadrants@normal}[4]{
  \renewcommand\arraystretch{1.5}
   \begin{array}{c|c}
     #1 & #2 \\ \hline
     #3 & #4
   \end{array}}
\newcommand{\quadrants@small}[4]{
  \renewcommand\arraystretch{0.9}
   \begin{array}{@{~}c@{~}|@{~}c@{~}}
     \scriptstyle #1 & \scriptstyle #2 \\ \hline
     \scriptstyle #3 & \scriptstyle #4
   \end{array}}
\newcommand\quadrants{\@ifstar\quadrants@small\quadrants@normal}
\makeatother

In addition, to allude to the reader's intuition, expressions of the form
$a/b/c/d$ will be written as $\quadrants*{a}{b}{c}{d}$ when the slices
represent quadrants.

\makeatletter
\newcommand{\lbox@small}[1]{ {\setlength{\fboxsep}{1pt}\fbox{\small #1}} }
\newcommand{\lbox@tiny}[1]{ {\setlength{\fboxsep}{1pt}\fbox{\tiny #1}} }
\newcommand\lbox{\@ifstar\lbox@tiny\lbox@small}
\makeatother

\begin{center}
\fbox{$\begin{array}{ll}
       \mbox{\underline{Slice}} \\ 
       f ~=~ \theta\,i\,j\mapsto \cdots \\
       X_1 ~=~ \_\times I_0\times J_0 &
       X_2 ~=~ \_\times I_0\times J_1 \\
       X_3 ~=~ \_\times I_1\times J_0 &
       X_4 ~=~ \_\times I_1\times J_1 \\[.5em]
       \cspan2{\mbox{\small ({\it recall that each} ``\_'' {\it is a fresh type variable})}}
       \end{array}$}
\end{center}

\begin{equation}
  \renewcommand\arraystretch{1.2}
  \begin{array}{@{}r@{}l@{}c@{}c@{}l@{}l@{}}
    A^{^{IJ}} =~ & \lspan5{\psi\mapsto \fix \quadrants{\lbox A}{\lbox B}{\lbox C}{\lbox D}} \\
	\lbox A ~=~ & \theta\,& i & j & \mapsto\min\,\langle~ & \psi_{ij} \\
	      & & ^{^{(I_0)}} & ^{^{(J_0)}} & & \min \vtyped p I \mapsto\theta_{pj}+w_{pij}, \\
	      & & & & & \min \vtyped q J \mapsto\theta_{iq}+w'_{qji} ~\rangle \\
	\lbox B ~=~ & \theta\,& i & j & \mapsto\min\,\langle~ & \psi_{ij} \\
	      & & ^{^{(I_0)}} & ^{^{(J_1)}} & & \min \vtyped p I \mapsto\theta_{pj}+w_{pij}, \\
	      & & & & & \min \vtyped q J \mapsto\theta_{iq}+w'_{qji} ~\rangle \\
	\lbox C ~=~ & \theta\,& i & j & \mapsto\min\,\langle~ & \psi_{ij} \\
	      & & ^{^{(I_1)}} & ^{^{(J_0)}} & & \min \vtyped p I \mapsto\theta_{pj}+w_{pij}, \\
	      & & & & & \min \vtyped q J \mapsto\theta_{iq}+w'_{qji} ~\rangle \\
	\lbox D ~=~ & \theta\,& i & j & \mapsto\min\,\langle~ & \psi_{ij} \\
	      & & ^{^{(I_1)}} & ^{^{(J_1)}} & & \min \vtyped p I \mapsto\theta_{pj}+w_{pij}, \\
	      & & & & & \min \vtyped q J \mapsto\theta_{iq}+w'_{qji} ~\rangle \\
  \end{array}
  \label{tactics:A sliced}
\end{equation}

\begin{tacticbox}{Let}
   e[\square] ~=~ \quadrants*{\square}{\lbox*B}{\lbox*C}{\lbox*D} \qquad
   t ~=~ \lbox A
\end{tacticbox}

\begin{equation}
  A^{^{IJ}} =~ \psi\mapsto \fix \left(\lbox A \applt z\mapsto\quadrants{z}{\lbox B}{\lbox C}{\lbox D}\right)
\end{equation}

\begin{tacticbox}{Stratify[with Padding]}
  \begin{array}{@{} l @{} l @{}}
    f ~=~ \quadrants*{\lbox*A}{\dot\psi}{\dot\psi}{\dot\psi}
         & \mbox{\small ({\it recall that } $\dot\psi=\theta\mapsto\psi$)} \\
    g ~=~ z\mapsto\quadrants*{z}{\lbox*{B}}{\lbox*C}{\lbox*D} &
    \qquad\psi=\psi
  \end{array}
\end{tacticbox}

\begin{equation}
  A^{^{IJ}} =~ \psi\mapsto \fix \quadrants{\lbox A}{\dot\psi}{\dot\psi}{\dot\psi} ~\applt~ \psi\mapsto\fix\quadrants{\dot\psi}{\lbox B}{\lbox C}{\lbox D}
\end{equation}

Notice that an existing variable $\psi$ is reused, rebinding any occurrences within $\lbox B$, $\lbox C$, $\lbox D$.
This effect is useful, as it limits the context of the expression: the inner $\psi$ shadows the outer $\psi$,
meaning $\lbox B$, $\lbox C$, $\lbox D$ do not need to access the data that was input to $\lbox A$, only its
output.

\medskip
The sequence Let, Stratify[with Padding] is now applied in the same manner to $\lbox B$
and $\lbox C$. We do not list the applications as they are analogous to the previous ones.

\begin{equation}
  \renewcommand\arraystretch{1.5}
  \begin{array}{l@{}l}
    A^{^{IJ}} =~ \psi\mapsto{} & \fix \quadrants{\lbox A}{\dot\psi}{\dot\psi}{\dot\psi} ~\applt~ 
                 \psi\mapsto\fix\quadrants{\dot\psi}{\lbox B}{\dot\psi}{\dot\psi} ~\applt \\
               & \psi\mapsto\fix\quadrants{\dot\psi}{\dot\psi}{\lbox C}{\dot\psi} ~\applt~
                 \psi\mapsto\fix\quadrants{\dot\psi}{\dot\psi}{\dot\psi}{\lbox D}
  \end{array}
\end{equation}

\begin{figure}
\includegraphics[width=.47\textwidth]{img/arbiter-stratify2}\\
\includegraphics[width=.47\textwidth]{img/arbiter-stratify3}
\caption{
  Stratification steps for phase ``A'' of Simplified Arbiter.}
\end{figure}

\begin{tacticbox}{Synth}
	\begin{array}{@{}l@{}c@{}c@{}l@{}l}
       \lspan5{h_1= \lbox A} \\
       \lspan5{h_{2,3,4}=\dot\psi} \\
	   f_1 = \theta\,& i & j & \mapsto\min\,\langle~ & \psi_{ij} \\
	      & ^{^{(I_0)}} & ^{^{(J_0)}} & & \min \vtyped p {I_0} \mapsto\theta_{pj}+w_{pij}, \\
	      & & & & \min \vtyped q {J_0} \mapsto\theta_{iq}+w'_{qji} ~\rangle \\
	   \lspan5{f_{2,3,4} = \dot\psi}
   \end{array}
\end{tacticbox}

\begin{equation}
  \renewcommand\arraystretch{1.5}
  \begin{array}{l@{}l}
    A^{^{IJ}} =~ \psi\mapsto{} & \quadrants{A^{^{I_0J_0}}_\psi\!\!\!}{\psi}{\psi}{\psi} ~\applt~ 
                 \psi\mapsto\fix\quadrants{\dot\psi}{\lbox B}{\dot\psi}{\dot\psi} ~\applt \\
               & \psi\mapsto\fix\quadrants{\dot\psi}{\dot\psi}{\lbox C}{\dot\psi} ~\applt~
                 \psi\mapsto\fix\quadrants{\dot\psi}{\dot\psi}{\dot\psi}{\lbox D}
  \end{array}
  \label{fix A}
\end{equation}

We note that $\fix f_1=A^{^{I_0J_0}}$ are {\bf identical} terms. Also, we took the liberty
to simplify $\fix\dot\psi$ into $\psi$ --- although this is not necessary --- just to display
a shorter term.

\medskip
The next few tactics will focus on the subterm $\lbox B$ from \eqref{tactics:A sliced}.

\begin{equation}
  \renewcommand\arraystretch{1.2}
  \begin{array}{@{}r@{}l@{}c@{}c@{}l@{}l@{}}
	\lbox B ~=~ & \theta\,& i & j & \mapsto\min\,\langle~ & \psi_{ij} \\
	      & & ^{^{(I_0)}} & ^{^{(J_1)}} & & \min \vtyped p I \mapsto\theta_{pj}+w_{pij}, \\
	      & & & & & \min \vtyped q J \mapsto\theta_{iq}+w'_{qji} ~\rangle
  \end{array}
\end{equation}

\begin{tacticbox}{Slice}
  \begin{array}{@{} l @{}}
    f = \min \vtyped q J \mapsto \theta_{iq}+w'_{qji} \\
    X_1 = J_0\to\_ \qquad X_2 = J_1\to\_
  \end{array}
\end{tacticbox}

\begin{equation}
  \renewcommand\arraystretch{1.2}
  \begin{array}{@{}r@{}l@{}c@{}c@{}l@{}l@{}l@{}}
	\lbox B ~=~ & \theta\,& i & j & \mapsto\min\,\big\langle~ & \psi_{ij} \\
	      & & ^{^{(I_0)}} & ^{^{(J_1)}} & & \lspan2{\min \vtyped p I \mapsto\theta_{pj}+w_{pij},} \\
	      & & & & & \min \big( & (\vtyped q {J_0} \mapsto\theta_{iq}+w'_{qji}) ~\big/~ \\
	      & & & & & & (\vtyped q {J_1} \mapsto\theta_{iq}+w'_{qji})\big)  ~\big\rangle
  \end{array}
\end{equation}

\begin{tacticbox}{Distributivity}
  \begin{array}{@{} l @{}}
    e[\square] = \min\square \\
    t_1 = \min \vtyped q {J_0} \mapsto \theta_{iq}+w'_{qji} \\
    t_2 = \min \vtyped q {J_1} \mapsto \theta_{iq}+w'_{qji} \\
  \end{array}
\end{tacticbox}

\begin{tacticbox}{Associativity}
  \begin{array}{@{} l @{} l @{}}
    \lspan2{\reduce = \min} \\
    \overline x_1 ={} & \psi_{ij} \\
    \overline x_2 ={} & \min \vtyped p I \mapsto\theta_{pj}+w_{pij} \\
    \overline x_3 ={} & \min \vtyped q {J_0} \mapsto \theta_{iq}+w'_{qji} ~, \\
                      & \min \vtyped q {J_1} \mapsto \theta_{iq}+w'_{qji}
  \end{array}
\end{tacticbox}

\begin{equation}
  \renewcommand\arraystretch{1.2}
  \begin{array}{@{}r@{}l@{}c@{}c@{}l@{}l@{}}
	\lbox B ~=~ & \theta\,& i & j & \mapsto\min\,\big\langle~ & \psi_{ij} \\
	      & & ^{^{(I_0)}} & ^{^{(J_1)}} & & \min \vtyped p I \mapsto\theta_{pj}+w_{pij}, \\
	      & & & & & \min \vtyped q {J_0} \mapsto\theta_{iq}+w'_{qji}, \\
	      & & & & & \min \vtyped q {J_1} \mapsto\theta_{iq}+w'_{qji} ~\big\rangle
  \end{array}
\end{equation}

\begin{tacticbox}{Let[$\reduce$]}
  \begin{array}{@{} r @{} l @{}}
    e[\square] ={} & \quadrants{\dot\psi}{\theta\,i\,j\mapsto\square}{\dot\psi}{\dot\psi} \\
    \overline a ={} & \psi_{ij}, ~\min \vtyped q{J_0}\mapsto \theta_{iq}+w'_{qji} \\
    \overline b ={} & \min \vtyped p I \mapsto\theta_{pj}+w_{pij}, \\
                    & \min \vtyped q {J_1} \mapsto\theta_{iq}+w'_{qji}
  \end{array}
\end{tacticbox}

\begin{equation}
  \renewcommand\arraystretch{1.2}
  \begin{array}{r @{} l @{} c @{} c @{} l @{} l}
    \lspan6{
    \quadrants{\dot\psi}{\lbox B}{\dot\psi}{\dot\psi} =
      \fix\left(\lbox E \applt z\mapsto\quadrants{\dot\psi}{\lbox F}{\dot\psi}{\dot\psi}\right) 
    } \\[1.5em]
    ~\lbox E ={} &
      \theta & i & j & \mapsto\min\langle & \psi_{ij}, \\
             & & ^{^{(I_0)}} & ^{^{(J_1)}} &
                                          & \min \vtyped q{J_0}\mapsto\theta_{iq}+w'_{qji} \rangle \\
    ~\lbox F ={} &
      \theta & i & j & \mapsto\min\langle & z_{\theta ij}, \\
             & & ^{^{(I_0)}} & ^{^{(J_1)}} &
                                         & \min \vtyped p I \mapsto\theta_{pj}+w_{pij}, \\
             & & & &                     & \min \vtyped q {J_1} \mapsto\theta_{iq}+w'_{qji}\rangle
  \end{array}
\end{equation}

\begin{tacticbox}{Stratify[with Padding]}
  \begin{array}{@{} l @{}}
    f ~=~ \quadrants*{\dot\psi}{\lbox*E}{\dot\psi}{\dot\psi} \\
    g ~=~ z\mapsto\quadrants*{\dot\psi}{\lbox*F}{\dot\psi}{\dot\psi}
    \qquad\psi=\psi
  \end{array}
\end{tacticbox}

\begin{equation}
  \renewcommand\arraystretch{1.2}
  \begin{array}{r @{} l @{} c @{} c @{} l @{} l}
    \lspan6{
    \fix\quadrants{\dot\psi}{\lbox B}{\dot\psi}{\dot\psi} =
      \fix\quadrants{\dot\psi}{\lbox E}{\dot\psi}{\dot\psi} \applt
      \psi\mapsto\fix\quadrants{\dot\psi}{\lbox F}{\dot\psi}{\dot\psi}
    } \\[1.5em]
    ~\lbox E ={} &
      \theta & i & j & \mapsto\min\langle & \psi_{ij}, \\
             & & ^{^{(I_0)}} & ^{^{(J_1)}} &
                                          & \min \vtyped q{J_0}\mapsto\theta_{iq}+w'_{qji} \rangle \\
    ~\lbox F ={} &
      \theta & i & j & \mapsto\min\langle & \psi_{ij}, \\
             & & ^{^{(I_0)}} & ^{^{(J_1)}} &
                                          & \min \vtyped p I \mapsto\theta_{pj}+w_{pij}, \\
             & & & &                      & \min \vtyped q {J_1} \mapsto\theta_{iq}+w'_{qji}\rangle
  \end{array}
\end{equation}

\noindent
Define
\begin{equation}
  \renewcommand\arraystretch{1.2}
  \begin{array}{@{}l @{} l @{\!} c @{} c @{} l @{} l@{}}
  B^{^{IJ_0J_1}} =~ & \psi\mapsto \\
      & \fix
      \theta & i & j & \mapsto\min\langle & \psi_{ij}, \\
           & & ^{^{(I)}} & ^{^{(J_1)}} &
                                          & \min \vtyped q{J_0}\mapsto\theta_{iq}+w'_{qji} \rangle
  \end{array}
\end{equation}

\begin{tacticbox}{Synth}
  \begin{array}{@{} l @{} c @{} c @{} l @{} l @{}}
    \lspan5{h_2 = \lbox E} \\
    \lspan5{h_{1,3,4} = \dot\psi} \\
    f_2 = 
      \theta & i & j & \mapsto\min\langle & \psi_{ij}, \\
             & ^{^{(I_0)}} & ^{^{(J_1)}} &
                                          & \min \vtyped q {J_0} \mapsto\theta_{iq}+w'_{qji}\rangle \\
    \lspan5{f_{1,3,4} = \dot\psi}
  \end{array}
\end{tacticbox}

\begin{tacticbox}{Synth}
  \begin{array}{@{} l @{} c @{} c @{} l @{} l @{}}
    \lspan5{h_2 = \lbox F} \\
    \lspan5{h_{1,3,4} = \dot\psi} \\
    f_2 = 
      \theta & i & j & \mapsto\min\langle & \psi_{ij}, \\
             & ^{^{(I_0)}} & ^{^{(J_1)}} &
                                          & \min \vtyped p {I_0} \mapsto\theta_{pj}+w_{pij}, \\
             & & &                        & \min \vtyped q {J_1} \mapsto\theta_{iq}+w'_{qji}\rangle \\
    \lspan5{f_{1,3,4} = \dot\psi}
  \end{array}
\end{tacticbox}

\begin{equation}
  \fix\quadrants{\dot\psi}{\lbox B}{\dot\psi}{\dot\psi} ~=~
    \quadrants{\psi}{B^{^{I_0J_0J_1}}_\psi\!\!\!\!}{\psi}{\psi} ~\applt~
    \psi\mapsto\quadrants{\psi}{A^{^{I_0J_1}}_\psi\!\!\!}{\psi}{\psi}
  \label{fix B}
\end{equation}

\medskip\noindent
In a similar manner, we will obtain the following:

\begin{equation}
  \fix\quadrants{\dot\psi}{\dot\psi}{\lbox C}{\dot\psi} ~=~
    \quadrants{\psi}{\psi}{C^{^{I_0I_1J_0}}_\psi\!\!\!\!}{\psi} ~\applt~
    \psi\mapsto\quadrants{\psi}{\psi}{A^{^{I_1J_0}}_\psi\!\!\!}{\psi}
  \label{fix C}
\end{equation}

\begin{equation}
  \renewcommand\arraystretch{1.2}
  \begin{array}{@{}l @{} l @{\!} c @{} c @{} l @{} l@{}}
  C^{^{I_0I_1J}} =~ & \psi\mapsto \\
      & \fix
      \theta & i & j & \mapsto\min\langle & \psi_{ij}, \\
           & & ^{^{(I_1)}} & ^{^{(J)}} &
                                          & \min \vtyped p {I_0} \mapsto\theta_{pj}+w_{pij} \rangle
  \end{array}
\end{equation}

\medskip\noindent
And\coa{Do we want to expand these?} ---

\begin{equation}
  \begin{array}{@{} l @{} l @{}}
    \fix\quadrants{\dot\psi}{\dot\psi}{\dot\psi}{\lbox D} ~=~ &
      \quadrants{\psi}{\psi}{\psi}{B^{^{I_1J_0J_1}}_\psi\!\!\!\!} ~\applt~
      \psi\mapsto\quadrants{\psi}{\psi}{\psi}{C^{^{I_0I_1J_1}}_\psi\!\!\!} \\
    &
       ~\applt~ \psi\mapsto\quadrants{\psi}{\psi}{\psi}{A^{^{I_1J_1}}_\psi\!\!\!}
  \end{array}
  \label{fix D}
\end{equation}

This gives the stratified version as shown in \Cref{evaluation:arbiter stratify A chain}.
The read and write regions are already encoded in the types of $A$, $B$, $C$ in 
\eqref{fix A}, \eqref{fix B}, \eqref{fix C}, and \eqref{fix D}.

\hrule
\bigskip


\subsection{Soundness}

\renewenvironment{proof}{\noindent{\bf Proof.~}}{}

\begin{theorem}
Let $s=s'$ be an instance of one of the tactics introduced in this section.
let $a_i=b_i$, $i=1..k$, be the proof obligations. If $\semp{a_i}=\semp{b_i}$
for all interpretations of the free variables of $a_i$ and $b_i$, then
$\semp{s}=\semp{s'}$ for all interpretations of the free variables of $s$ and $s'$.
\end{theorem}

\begin{proof}
For the tactics with {\bf Obligations:} tactic, the theorem is trivial.

\medskip
For Stratify, let $f$, $g$ be partial functions such that
\[\renewcommand\arraystretch{1.3}
  \forall \theta,\zeta.\quad \begin{array}{l}f\,(g\,\zeta\,\theta) ~=~ f\,\zeta \\
  g\,(f\,\theta)\,\theta ~=~ h\,\theta
  \end{array}\quad\]
  
Assume that $\zeta = \fix f$ and $\theta = \fix (g\,\zeta)$. That is,
\[\renewcommand\arraystretch{1.3}
  \begin{array}{l}
    f\,\zeta = \zeta\\
    g\,\zeta\,\theta = \theta
  \end{array}\]
  
Then ---
\[\renewcommand\arraystretch{1.3}
  \begin{array}{l@{}l}
   h\,\theta & {}= g\,(f\,\theta)\,\theta = g\,(f\,(g\,\zeta\,\theta))\,\theta = \\
             & {}= g\,(f\,\zeta)\,\theta = \theta
  \end{array}\]
  
So $\theta = \fix h$. We get $\fix h = \fix \big(g \,(\fix f)\big)$, or, equivalently,
\[\fix h = \fix f \applt \psi\mapsto\fix (g\,\psi)\]

Now instantiate $h$, $f$, and $g$, with $f\applt g$, $f$, and $g'$ from \Cref{tactics:Stratify},
and we obtain the equality in the tactic.

\medskip
For Synth, assume
\[\renewcommand\arraystretch{1.3}
  \forall \overline\theta.\quad h\,(f\,\overline\theta)=f\,\overline\theta \quad\]

And let $\theta=\theta_{1..r}$ such that $\theta_i=\fix f_i$. So $f_i\,\theta_i=\theta_i$.
Let $\theta=\theta_1::\T_1/\cdots/\theta_r::\T_r$.
\[\renewcommand\arraystretch{1.3}
  \begin{array}{l@{}l}
   f\,\overline\theta & {}= (f_1\,\theta_1)::\T_1 / \cdots / (f_r\,\theta_r)::\T_r =  \\
     & {}= \theta_1::\T_1 / \cdots / \theta_r::\T_r = \theta \\[.5em]
   h\,\theta & {}= h\,(f\,\overline\theta) = f\,\overline\theta = \theta
   \end{array}\qquad\]
   
Then $\theta=\fix h$;\\
We get $\fix h = (\fix f_1)::\T_1 / \cdots / (\fix f_r)::\T_r$,
as required.
\end{proof}
